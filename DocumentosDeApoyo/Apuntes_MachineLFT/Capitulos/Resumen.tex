The ordering of words in a document provides a semantic meaning that cannot be
inferred from a representation based on only the frequencies of words in that document.
(Pg. 23)

1. Text as a bag-of-words: This is the most commonly used representation for text mining.
In this case, the ordering of the words is not used in the mining process. The set
of words in a document is converted into a sparse multidimensional representation,
which is leveraged for mining purposes. Therefore, the universe of words (or terms)
corresponds to the dimensions (or features) in this representation. For many applications
such as classification, topic-modeling, and recommender systems, this type of
representation is sufficient. (Pg.23)

2. Text as a set of sequences: In this case, the individual sentences in a document are
extracted as strings or sequences. Therefore, the ordering of words matters in this
representation, although the ordering is often localized within sentence or paragraph
boundaries. A document is often treated as a set of independent and smaller units (e.g.,
sentences or paragraphs). This approach is used by applications that require greater
semantic interpretation of the document content. This area is closely related to that
of language modeling and natural language processing. The latter is often treated as a
distinct field in its own right.(Pg.24)


It is important to be cognizant of the sparse and high-dimensional characteristics of text
when treating it as a multidimensional data set. This is because the dimensionality of the
data depends on the number of words which is typically large. Furthermore, most of the word
frequencies (i.e., feature values) are zero because documents contain small subsets of the
vocabulary. Therefore, multidimensional mining methods need to be cognizant of the sparse
and high-dimensional nature of the text representation for best results (pg. 24)


In fact, some models, such as the linear support vector machines
discussed in Chap. 6, are inherently suited to sparse and high-dimensional data(Pg. 24)

A data set corresponds to a collection of documents, which is also referred to as
a corpus. The complete and distinct set of words used to define the corpus is also referred
to as the lexicon.(pg. 24)


text is a high-dimensional, sparse, and non-negative representation.(24)


página 25, párrafo 1