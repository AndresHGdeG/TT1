\chapter{Introducción}


%--------------------------------------Angeda-------------------------------------------%

\section{Agenda}

\begin{multicols}{2}

	\begin{enumerate}

		\item \textbf{Problemática}
		\item \textbf{Justificación}
		\item \textbf{Solución propuesta}
		\item \textbf{Objetivo general}
		\item \textbf{Objetivos específicos}
		\item \textbf{Estado del arte}
		\item \textbf{Marco teórico}
		\item \textbf{Análisis y diseño}
		\item \textbf{Pruebas y resultados}
		\item \textbf{Trabajo para TT2}
		\item \textbf{Referencias}

	\end{enumerate}

\end{multicols}


%---------------------------------------Problemática-------------------------------------------%
\section{Introducción}

Importancia de las noticias



\section{Problemática}

\begin{enumerate}
	

	\item Clasificación de noticias
	\begin{itemize}
		\item Sitios web de noticias con secciones variadas (Sinonimos para los tópicos)
		\item Sitios web sin secciones definidas
	\end{itemize}
	
	\item  Métodos de recopilación de la información en Internet \footnote{No mencionar si se puede o no extraer informaciòn de un sitio a menos que nos preguntens}

	\begin{itemize}
		\item Manual (Tiempo, dinero y esfuerzo)
		\item Crawler (Depende de las etiquetas definidas)

	\end{itemize}

% Poli genereba gaceta de noticias y tenia equipo el cual hacía la clasificación 
 

\end{enumerate}

%----------------------------------Objetivo general-------------------------------------------%
\section{Objetivo general}

  Crear un recolector de noticias (UTILIZANDO UN CRAWLER), el cual permita extrawe información de diferentes fuentes de Internet, mediante la cración de un crawler(web scraping y web creawling), para su posterior clasificación mediante análisis automático del conteniddo, para mostrar aquellas noticias que satisfagan los filtros establecidos por el usuario.

  \begin{itemize}
	\item Crear un recoleclector de noticias
	\item Clasificar noticias
	\item Mostrar resultados
	\end{itemize}

%Comentarle a los sinodales que se tenia definidos foros, blogs etc
%---------------------------------------Objetivo Específico------------------------------------%

\section{Objetivos Específicos}

\begin{itemize}

  \item Desarrollar un recolector de noticias, el cual permita obtener información de diferentes fuentes como diarios, sitios de noticias, blogs y foros

  \item Analizar de forma automática el contenido de las noticias para satisfacer los filtros establecidos por el usuario

  \item Mostrar las noticias que cumplieron con los filtros establecidos, así como su enlace (URL) para redirigirlos a la página de la noticia

  \item Afinar el clasificador de noticias realizado en el trabajo terminal 2017-A02 para utilizarlo en el contexto de esta propuesta (filtro de sección) 

\end{itemize}
%--------------------------------------Justificación------------------------------------------%
\section{Justificación}

Según el díario El Economista \citep{SU1} el sitio web Animal Político\footnote{www.animalpolitico.com}
ocupa el lugar número cuatro en el ranking de medios nativos digitales, clasifica sus noticias de una manera poco habitual para los lectores como la sección
\textbf{El sabueso}, \textbf{El plumaje}, \textbf{Hablemos de . . . }, entre otras, lo que hace complicado obtener los artículos con los métodos tradicionales 
de recopilación que, se basan sólo en las etiquetas que identifican cada sección y no el contenido de las noticias.


	1. Dependencia de etiquetadas dadas por el sitio web
	2. Apesar de la existencia de las etiquetas no son homogeneos,las etiquetas varían (Una etique es nombre asignado en XML a un campo)

\section{Dudas de presentación}
 
\begin{itemize}

	\item 1 Costo del algoritmo
	\item 2 Tiempo de extracción
	\item 3 9 páginas cuanto tiempo extraemos
	\item 4 Número de palabras



	\item Para el corpus 
	\item 	1,000 por sección
	\item 		800 Para el enetrenamiento
	\item 		200 Para probar el resultado

	\item NO sobreentrenar a un algoritmo

	\item Considerar Yahoo



	\item crawler
	\item web scraping
	\item web crawling

	\item (Medir número de peticiones por noticia)
	\item (Tiempo ocupado para extrer noticias)

\end{itemize}

\section{Agregar al marco teórico}

	\begin{itemize}
		\item Xpath
		\item Árbol generado por la página
		\item importancia de XML ()
	\end{itemize}	

