\chapter{Introducción}


%--------------------------------------Angeda-------------------------------------------%

\section{Agenda}

\begin{multicols}{2}

	\begin{enumerate}

		\item \textbf{Problemática}
		\item \textbf{Justificación}
		\item \textbf{Solución propuesta}
		\item \textbf{Objetivo general}
		\item \textbf{Objetivos específicos}
		\item \textbf{Estado del arte}
		\item \textbf{Marco teórico}
		\item \textbf{Análisis y diseño}
		\item \textbf{Pruebas y resultados}
		\item \textbf{Trabajo para TT2}
		\item \textbf{Referencias}

	\end{enumerate}

\end{multicols}


%---------------------------------------Problemática-------------------------------------------%
\section{Problemática}

\begin{enumerate}
	
	\item Etiquetar datos cuesta tiempo dinero y esfuerzo
	
\end{enumerate}

%----------------------------------Objetivo general-------------------------------------------%
\section{Objetivo general}

  Crear un recolector de noticias, el cual permita recopilar información de diferentes fuentes como diarios, sitios de noticias, foros y mediante el 
  análisis automático de su contenido muestre aquellas noticias que satisfagan los filtros establecidos por el usuario.


%---------------------------------------Objetivo Específico------------------------------------%

\section{Objetivos Específicos}

\begin{itemize}

  \item Desarrollar un recolector de noticias, el cual permita obtener información de diferentes fuentes como diarios, sitios de noticias, blogs y foros
  \item Analizar de forma automática el contenido de las noticias para satisfacer los filtros establecidos por el usuario
  \item Mostrar las noticias que cumplieron con los filtros establecidos, así como su enlace (URL) para redirigirlos a la página de la noticia
  \item Afinar el clasificador de noticias realizado en el trabajo terminal 2017-A02 para utilizarlo en el contexto de esta propuesta (filtro de sección) 

\end{itemize}
%--------------------------------------Justificación------------------------------------------%
\section{Justificación}


Según el díario El Economista \citep{SU1} el sitio web Animal Político\footnote{www.animalpolitico.com}
ocupa el lugar número cuatro en el ranking de medios nativos digitales, clasifica sus noticias de una manera poco habitual para los lectores como la sección
\textbf{El sabueso}, \textbf{El plumaje}, \textbf{Hablemos de . . . }, entre otras, lo que hace complicado obtener los artículos con los métodos tradicionales 
de recopilación que, se basan sólo en las etiquetas que identifican cada sección y no el contenido de las noticias.

