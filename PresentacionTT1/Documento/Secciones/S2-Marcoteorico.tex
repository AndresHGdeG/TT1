\chapter{Marco teórico}
Mostrar el diagrama de los pilares del marco teórico


%-----------------------------------Inteligencia artificial------------------------------------%
\section{Inteligencia artificial}
    Son muchas las definiciones que se encuentran de la inteligencia artificial o
    IA, en sus inicios se propone como las actividades asociadas al pensamiento
    humano, tareas como, toma de decisiones, resolución de problemas y aprendizaje \citep{SD1}    

%-----------------------------------Aprendizaje automático------------------------------------%
\section{Aprendizaje automático}
    El Aprendizaje Automático es una rama de la Inteligencia Artificial; permite desarrollar 
    algoritmos que tienen la capacidad de predecir los cambios que se acontecen en una tarea específica \citep{SD2}.\\
    \begin{itemize}
        \item \textbf{Aprendizaje supervisado}
        Se proporciona un conjunto de datos de
        entrenamiento con las respuestas correctas y, con base a este conjunto
        de entrenamiento, el algoritmo se generaliza para responder correctamente 
        a todas las entradas posibles.
        %-HACER ENFÁSIS EN QUE NOSOTROS REALIZAREMOS APRENDIZAJE SUPERVISADO
        %Analisis de patrones con un con base en un conjunto de datos previamente definido
        \item \textbf{Aprendizaje no supervisado}
        No se proporcionan datos de entrenamiento, el algoritmo intenta identificar 
        similitudes entre las entradas para clasificar en conjuntos.    
        %Reconocimiento de patrones
    \end{itemize}

%-----------------------------------Aprendizaje automático------------------------------------%
\section{Preprocesamiento}
    El Pre-procesamiento es necesario para convertir el formato no estructurado en un formato estructurado \citep{SD1}.
    A menudo el texto contiene información extraña como etiquetas, \textit{anchor text}\footnote{Es el texto mostrado en 
    los enlaces o hipervínculos, \textbf{Texto de anclaje} en español.}, y otras características. En muchos casos las palabras 
    son variaciones de otros (sinónimos) por el tipo de redacción, el contexto, para eliminar redundancia. Algunas palabras simplemente 
    tienen faltas de ortografía. El proceso de convertir una secuencia de caracteres en una secuencia de palabras (tokens), es llamado ``Tokenización''.

\subsection{Texto como bolsa de palabras}
    Es la representación mas común. No se contempla el orden de las palabras el proceso. El conjunto de palabras en el documento se convierten en 
    \textit{Sparse multidimentional reprentation}, el cual corresponde a la dimensión en esta representación. Se utiliza para la clasificación, 
    sistemas de recomendación.

%-----------------------------------Algoritmos------------------------------------%
\section{Algoritmos}

%Realizar una tabla comparativa con los 4 algoritmos 


\subsection{Matriz de confusión}
    Una matriz de confusión es una representación de la información de los resul-
    tados obtenidos por un clasificador, dicha matriz suele ser de tamaño n x n,
    donde n es es el número de clases diferentes con las que se están trabajando
    (Visa et al., 2011).

    %Nos permite medir que tan bueno es un algoritmo


%----------------------------------Recoleccion de noticias---------------------------------%
\section{Recoleccion de noticias}

\subsection{Web Scraping}
    La recopilación de datos de Internet es una técnica que se realiza de manera
    manual, sin embargo el Web Scraping es el conjunto de técnicas utilizadas
    para obtener de manera automática información de un sitio web.

    %Etapas del web crawling

\subsection{Crawler}
