%-----------Eliminamos la identación de los párrafos-------%
\setlength{\parindent}{0pt}


%-------- Se define las lineas del pie y encabezado de de página
\fancypagestyle{plain}{



  %-------------Imagen de pie de página pares---------%

    \fancyfoot[LE]{\begin{picture}(0,0) \put(-170,-80){ \includegraphics[scale=0.7]{Imagenes/Foot-blue.png}}\end{picture}}

  %-------------Imagen de pie de página impares---------%
    \fancyfoot[LO]{\begin{picture}(0,0) \put(-125,-80){ \includegraphics[scale=0.7]{Imagenes/Foot-blue.png}}\end{picture}}
	
  \renewcommand{\headrulewidth}{0.5pt}
  \renewcommand{\footrulewidth}{0.5pt}	
}

\pagestyle{plain}% aplicamos el estilo para todas las páginas


%------------------Definimos los colores a usar----------------%
\definecolor{myBlue}{HTML}{166995}
\definecolor{myDarkBlue}{HTML}{0D547B}
\definecolor{myGinda}{HTML}{921010}
\definecolor{myBlueRef}{HTML}{0277bd}
\definecolor{myBlueChapter}{HTML}{004784}


%---------Definimos el color de cada vínculo en el documento como links, urls, citas, etc.
\hypersetup{
    colorlinks=true,
    linkcolor=myDarkBlue,
    filecolor=magenta,      
    urlcolor=myBlue,
    citecolor=myBlueRef,
    pdftitle={Sharelatex Example}
}
