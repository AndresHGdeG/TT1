

\chapter{Bibliografía}

\section{Referencias}


[1] Importancia de las Noticias. 07/09/2018, de INNOVACION INTERNAUTICA Sitio web: https://innovainternetmx.com/2014/12/importancia-de-las-noticias/ \\

[2] Manning, C., Raghavan, P. and Schütze, H. (2009). Introduction to information retrieval. New York: Cambridge University Press, pp.443-459.
/los-10-periodicos-mas-influyentes-en-mexico/\\

[3] Eleconomista.com.mx. (2019). Ranking de Medios Nativos Digitales | El Economista. Sitio web: https://www.eleconomista.com.mx/Ranking-de-Medios-Nativos-Digitales.\\

[4] A. Téllez-Valero, M. Montes, O. F.-C. Gómez, y L. Villaseñor-Pineda, “Clasificación automática de textos de desastres naturales en méxico.”\\

[5] J. Fernando Sánchez Vega, "Clasificación de Texto Mediante Atributos Probabilísiticos de Concurrencia de Palabras."\\

[6] J. Cárdenas, G. Olivares y R. Alfaro, "Clasificación automática de textos usando redes de palabras."\\

[7] D. Ramdass and S. Seshasai, “Document classification for newspaper articles,” Document classification for newspaper articles, 2009.\\

[8] "Cloud Natural Language  |  Cloud Natural Language API  |  Google Cloud", Google Cloud, 2019. Sitio web: https://cloud.google.com/natural-language/?hl=Es-419.\\

[9] Fernando Macia, Googlebot, Human Level. Sitio web: https://www.humanlevel.com/diccionario-marketing-online/googlebot\\

[10] G. Nieto Fernández, Reconocimiento del Lenguaje, IBM. Sitio web: https://www.ibm.com/blogs/think/es-es/2017/05/16/watson-nlc-en-hogwarts/\\

[11] C. Olston and M. Najork, Foundations and TrendsR©inInformation RetrievalVol. 4, No. 3 (2010) 175–246, "Web Crawling".\\

[12] Comunidad de Madrid, www.emprendelo.es, "¿Qué es una página web?".\\

[13] E. Bruguera Payá. Universitat Oberta de Catalunya, "¿Qué es un blog?"\\

[14] Universidad Luterana Salvadoreña, "foros.pdf"\\

[15] M. Molina Vives,"Desarrollo del lenguaje"\\

[16] Russell, S. y Norvig, Prentice Hall, 2010, "P.Artificial Intelligence (A ModernApproach)".