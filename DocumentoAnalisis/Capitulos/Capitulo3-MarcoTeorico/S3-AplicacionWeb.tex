\MTtitle{Aplicación web}


\section{Sitios web}
Un sitio web \citep{CT15} es un conjunto de páginas web interconectadas y de acceso público que comparten 
un solo nombre de dominio. Los sitios web pueden ser creados y mantenidos por un individuo, grupo, empresa 
u organización para cumplir una variedad de propósitos. Todos estos sitios constituyen la World Wide Web. 

\subsection{Página web}
Una página web es un documento electrónico el cual forma parte de la WWW (\textit{World Wide Web}) generalmente 
construido en el lenguaje HTML (\textit{Hyper Text Markup Language}). Este documento puede contener enlaces que nos 
direcciona a otra página web. Para visualizar una página web es necesario de un browser o un navegador \citep{CT16}. 
Dentro de las páginas web podemos encontrar un sinfin de sitios los cuales pueden ser de nuestro interés.

\subsection{Blog}
Un blog es una página web en la cual el usuario no necesita conocimientos específicos del medio electrónico ni del 
formato digital para poder aportar contenidos de forma inmediata, ágil y constante desde cualquier punto de conexión 
a Internet \citep{CT17}. En un blog el usuario puede compartir cualquier tipo de información que sea de su agrado, 
teniendo una mayor libertad de expresión lo cual permite que otras personas compartan y comenten su manera de expresarse.

\subsection{Foro}
Un foro es una herramienta de comunicación asíncrona. Los foros permiten la comunicación de los participantes desde 
cualquier lugar en el que  esté  disponible  una  conexión  a Internet  sin  que  éstos  tengan  que  estar dentro del 
sistema al mismo tiempo, de ahí su naturaleza asíncrona. Brindando una mayor interacción entre distintos 
participantes y permitiendo conocer la opinión sobre un tema de distintas personas.

