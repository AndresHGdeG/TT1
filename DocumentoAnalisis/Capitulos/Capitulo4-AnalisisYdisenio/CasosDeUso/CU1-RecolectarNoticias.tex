\Tlabel{CU1}\Tsubsection{CU1 Recolectar noticias}

%================================Intriodcucción==================================%
%----------------------Resumen-----------------------------------%
\begin{large}
	\textbf{Resumen}\\
\end{large}

Brinda al usuario un punto de acceso para elegir una sección; las clasificaciones son, \textbf{Ciencia y tecnología}, \textbf{Política}, \textbf{Deportes}, \textbf{Economía} y  \textbf{Cultura}, posteriormente se recolectan noticias de la web, tomando como punto de partida los sitios establecidos previamente. Se crea un proceso de recolección independiente por cada sitio web, para simular un ambiente de extracción en paralelo; de cada sitio se recolectan las noticias publicadas; de cada artículo se obtiene \textbf{Fecha de publicación}, \textbf{Título}, \textbf{Contenido}, \textbf{URL de la noticia}, y de contar con ello el \textbf{Resumen}. Cabe destacar que las ligas contenidas en los sitios visitados son extraídas para su posterior análisis.\\

\begin{large}
	\textbf{Descripción}\\
\end{large} 

%=====================================Tabla 1====================================%


\begin{tabular}{|l|l|}
%-----------------------Ecanbezado-----------------------------------%
	\hline
	\multicolumn{1}{| >{\columncolor{black}}l|}{ \textcolor{myWhite}{\textbf{Caso de uso: }} }&
	\multicolumn{1}{| >{\columncolor{black}}l|}{ \textcolor{myWhite}{CU1 Recolectar noticias} }\\
	\hline
%-----------------------Actor----------------------------------------%
	\textbf{Actor:} & 	Usuario\\
	\hline

%-----------------------Propósito------------------------------------%

	\textbf{Propósito:} & Brindar una herramienta de recolección de noticias\\
	& de Internet(Crawler) \\
	\hline

%----------------------Entradas--------------------------------------%

	\textbf{Entradas:} & URL de las paginas por consultar\\
	\hline

%-----------------------Salidas--------------------------------------%

	\textbf{Salidas:} &$\bullet$ \Tref{MSG1}{MSG1 Catálogo vació}\\	
	&$\bullet$ \Tref{MSG3}{MSG3 Fallo en la recolección}\\
	\hline

%-----------------------Precondiciones-------------------------------%

	\textbf{Precondición:} & EL \textbf{Diccionario de URL'S} debe contener los\\
	&vínculos de los sitios a consultar\\
	\hline
%-----------------------Postcondiciones------------------------------%

	\textbf{Postcondiciones:} &$\bullet$  El usuario tendrá la facultad\\
	&\ \  de visualizar las noticias clasificadas\\
	&$\bullet$ El usuario podrá cambiar el periodo de búsqueda\\
	\hline

	%-----------------------Reglas de negocio----------------------------%

	\textbf{Reglas de negocio:} &$\bullet$ \RNref{RN1}{Número de palabras}\\
	&$\bullet$ \RNref{RN3}{Listado de fuentes noticiosas}\\
	&$\bullet$ \RNref{RN5}{Orden de publicación}\\
	&$\bullet$ \RNref{RN6}{Periodo de recolección}\\
	&$\bullet$ \RNref{RN7}{Campos recolectados de noticia}\\
	\hline

%---------------------------Errores----------------------------------%

%------Error 1----------%
	\textbf{Errores:} & $\bullet$ \TError{CU1}{Uno} Cuando no se ha recuperado \\
	&\ \ ninguna \textbf{dirección web} se muestra el mensaje\\
	&\ \  \Tref{MSG1}{MSG1 Catálogo vació}\\
	\hline

\end{tabular}


%=====================================Tabla 2====================================%
\begin{table}[H]

\begin{tabular}{|l|l|}
%-----------------------Ecanbezado-----------------------------------%
	\hline
	\multicolumn{1}{| >{\columncolor{black}}l|}{ \textcolor{myWhite}{\textbf{Caso de uso: }} }&
	\multicolumn{1}{| >{\columncolor{black}}l|}{ \textcolor{myWhite}{CU1 Recolectar noticias} }\\
	\hline
%------Error 3----------%
	 \textbf{Errores:}&$\bullet$ \TError{CU1}{Dos} Cuando no se puede extraer información de\\
	 &\ \  los sitios brindados, se muestra el mensaje \Tref{MSG3}{MSG3} \\
	 &\ \ \Tref{MSG3}{Fallo en la recolección}\\
	\hline

\end{tabular}\\\\
%\caption{ Tabla de descripción de CU1 }
\end{table}


%============================Trayectorias========================================%

%-----------------------Trayectoria Principal-----------------------%


\begin{large}
	\textbf{Trayectoria principal}\\
\end{large}	

\begin{enumerate}[1.]

	
	\item \actor Selecciona una opción de la pantalla \Tref{UI1}{UI1 Inicio}; \textbf{Política}, \textbf{Economía}, \textbf{Deportes}, \textbf{Ciencia y tecnología} o \textbf{Cultura}. 
	
	\item \sistema Obtiene las las direcciones web con base en la regla de negocio \RNref{RN3}{Listado de fuentes noticiosas}.
	
	\item \sistema Verifica que al menos se recupere una \textbf{Dirección web}.\TAref{CU1}{A}

	\item \sistema Muestra la pantalla \Tref{UI2}{Pantalla UI2 Espera de proceso}. \TAref{CU1}{B}	

	\item \sistema Verifica que no se haya recolectado noticias previamente. \TAref{CU1}{C}

	\item \sistema Por cada URL recuperada se extraen las noticias con base en la regla de negocio \RNref{RN6}{Periodo de recolección} y \RNref{RN7}{Campos recolectados de noticia.} \TAref{CU1}{D}

	\item \label{CU1:BuscarN}\sistema Incluye el caso de uso \textbf{CU2 Clasificar noticias}.

	\item \sistema Ordena las noticias clasificadas de acuerdo a la regla de negocio \RNref{RN5}{Orden de publicación.}

	\item \sistema \label{CU1:NoticiasR} Muestra la pantalla \Tref{UI3}{Pantalla UI3 Proceso concluido}.

	\item \finCU	

\end{enumerate}



%-------------------------trayectoria Alternativa A-----------------%
\begin{large}
	\Talterna{CU1}{A}\\
\end{large}	
\textbf{Condición:} \textit{No se ha recuperado ninguna dirección web \TEref{CU1}{Uno}}

\begin{enumerate}[{A-}1.]

	\item \sistema Muestra el mensaje \Tref{MSG1}{MSG1 Catálogo vació} en la pantalla \Tref{UI1}{UI1 Inicio}.

	\item \finCU

\end{enumerate}


%-------------------------trayectoria Alternativa B-----------------%
\begin{large}
	\Talterna{CU1}{B}\\
\end{large}	
\textbf{Condición:} \textit{El usuario ha presionado el botón cancelar}

\begin{enumerate}[{B-}1.]

	\item \actor Presiona el botón \textbf{Cancelar} de la pantalla \Tref{UI2}{Pantalla UI2 Espera de proceso}.

	\item \sistema Muestra la pantalla \Tref{UI1}{UI1 Inicio}.

	\item \finCU

\end{enumerate}


%-------------------------trayectoria Alternativa C-----------------%
\begin{large}
	\Talterna{CU1}{C}\\
\end{large}	
\textbf{Condición:} \textit{Ya se han recolectado noticias}

\begin{enumerate}[{C-}1.]

	\item \actor Continua en el paso \ref{CU1:NoticiasR} de la trayectoria principal.

	\item \finTA

\end{enumerate}

%-------------------------trayectoria Alternativa D-----------------%
\begin{large}
	\Talterna{CU1}{D}\\
\end{large}	
\textbf{Condición:} \textit{No se puede extraer información de los sitios web \TEref{CU1}{Dos}}

\begin{enumerate}[{D-}1.]

	\item \sistema Muestra el mensaje \Tref{MSG3}{MSG3 Fallo en la recolección} en la pantalla \Tref{UI1}{UI1 Inicio}.

	\item \finCU

\end{enumerate}



%================================Puntos de extención=============================%


\begin{large}
	\textbf{Puntos de extensión}\\
\end{large}	

%--------------------Puntos de extención 1------------------------%
\textbf{Causa de la extensión:} El usuario desea consultar las noticias clasificadas.\\
\textbf{Región de la trayectoria:} Proviene del paso \ref{CU1:NoticiasR} de la trayectoria principal.\\
\textbf{Extiende a :} \Tref{CU4}{CU4 Mostrar resultados}\\\\


