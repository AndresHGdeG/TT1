\Tlabel{CU1}\Tsubsection{CU1 Recolectar noticias}

%================================Intriodcucción==================================%
%----------------------Resumen-----------------------------------%
\begin{large}
	\textbf{Resumen}\\
\end{large}

Brinda al usuario un punto de acceso para elegir una sección; Las clasificaciones son, 
\textbf{Ciencia y técnología}, \textbf{Política}, \textbf{Deportes}, \textbf{Economía} y  
\textbf{Cultura}, posteriormente se recolectan noticias de la web, tomando como punto de 
partida los sitios establecidos previamente. Se crea un proceso de 
recolección independiente por cada sitio web, para simular un ambiente de extracción 
en paralelo; De cada sitio se recoletan las noticas publicadas; De cada artículo se obtiene 
\textbf{Fecha de publicación}, \textbf{Título}, \textbf{Contenido}, \textbf{URL de la noticia}, y de contar con ello el \textbf{Resumen}. Cabe destacar que 
las ligas contenidas en los sitios visitados son extraidas para su posterior análisis.\\

\begin{large}
	\textbf{Descripción}\\
\end{large} 

%=====================================Tabla 1====================================%

\begin{tabular}{|l|l|}
%-----------------------Ecanbezado-----------------------------------%
	\hline
	\multicolumn{1}{| >{\columncolor{black}}l|}{ \textcolor{myWhite}{\textbf{Caso de uso: }} }&
	\multicolumn{1}{| >{\columncolor{black}}l|}{ \textcolor{myWhite}{CU3 Recolectar noticias} }\\
	\hline
%-----------------------Actor----------------------------------------%
	\textbf{Actor:} & 	Usuario\\
	\hline

%-----------------------Propósito------------------------------------%

	\textbf{Propósito:} & Brindar una herramienta de recolección de noticias\\
	& de Internet(Crawler). \\
	\hline

%----------------------Entradas--------------------------------------%

	\textbf{Entradas:} & URL de las paginas por consultar\\
	\hline

%-----------------------Salidas--------------------------------------%

	\textbf{Salidas:} &$\bullet$ \Tref{MSG7}{MSG7 Petición vacía}\\	
	&$\bullet$ \Tref{MSG9}{MSG9 Fallo en la recolección}\\
	\hline

%-----------------------Precondiciones-------------------------------%

	\textbf{Precondición:} & EL \textbf{Diccionario de URL'S} debe contener los\\
	&vínculos de los sitios a consultar\\
	\hline
%-----------------------Postcondiciones------------------------------%

	\textbf{Postcondiciones:} &$\bullet$  El usuario tendrá la facultad\\
	&\ \  de visualizar las noticias clasificadas\\
	&$\bullet$ El usuario podrá cambiar el periodo de busqueda\\
	\hline
%-----------------------Reglas de negocio----------------------------%

	\textbf{Reglas de negocio:} &$\bullet$ \RNref{RN1}{Número de palabras}\\
	&$\bullet$ \RNref{RN3}{Diccionario de sitios}\\
	&$\bullet$ \RNref{RN5}{Orden de publicación}\\
	&$\bullet$ \RNref{RN6}{Número de noticias recolectadas}\\
	\hline

%---------------------------Errores----------------------------------%

%------Error 1----------%
	\textbf{Errores:} & $\bullet$ \TError{CU1}{Uno} Cuando no se ha recuperado \\
	&\ \ ninguna \textbf{dirección web} se muestra el mensaje\\
	&\ \  \Tref{MSG1}{MSG1 Catálago vacio}, fin del caso de uso\\

%------Error 2----------%	
	&$\bullet$ \TError{CU1}{Dos} Cuando no se ha encontrado noticias en \\
	&\ \ el día seleccionado se muestra el mensaje \Tref{MSG2}{MSG2}\\
	&\ \ \Tref{MSG2}{Petición vacía}, fin del caso de uso\\

\hline
\end{tabular}\\\\


%=====================================Tabla 2====================================%

\begin{tabular}{|l|l|}
%-----------------------Ecanbezado-----------------------------------%
	\hline
	\multicolumn{1}{| >{\columncolor{black}}l|}{ \textcolor{myWhite}{\textbf{Caso de uso: }} }&
	\multicolumn{1}{| >{\columncolor{black}}l|}{ \textcolor{myWhite}{CU3 Recolectar noticias} }\\
	\hline

%------Error 3----------%
	 \textbf{Errores:}&$\bullet$ \TError{CU1}{Tres} Cuando no se puede extraer información de\\
	 &\ \  los sitios brindados, se muestra el mensaje \Tref{MSG3}{MSG3} \\
	 &\ \ \Tref{MSG3}{Fallo en la recolección}, fin del caso de uso\\
	\hline
%-------------------------Autor--------------------------------------%
	\textbf{Autor:} & Carlos Andres Hernandez, Luis Daniel Meza\\
	\hline
\end{tabular}\\\\



%============================Trayectorias========================================%

%-----------------------Trayectoria Principal-----------------------%


\begin{large}
	\textbf{Trayectoria principal}\\
\end{large}	

\begin{enumerate}[1.]

	
	\item \actor Selecciona una opción de la pantalla \Tref{UI1}{UI1 Inicio}; \textbf{Política}, \textbf{Economía}, \textbf{Deportes}, \textbf{Ciencia y tecnología} o \textbf{Cultura}. 
	
	\item \sistema Obtiene las \textbf{Direcciones web}.
	
	\item \sistema Verifica que al menos se recupere una \textbf{Dirección web}. \TEref{CU1}{Uno}

	\item \sistema Muestra la pantalla \Tref{UI2}{Pantalla UI2 Espera de proceso}. \TAref{CU1}{A}	

	\item \sistema Verifica que no se haya recolectado noticias previamente. \TAref{CU1}{B}

	\item \sistema Por cada URL recuperada se extraen las noticias con base en la regla de negocio \RNref{RN6}{Número de noticias recolectadas}. \TEref{CU1}{Tres}

	\item \label{CU1:BuscarN}\sistema Incluye el caso de uso \Tref{CU2}{CU2 Clasificar noticias}.

	\item \sistema Obtiene la fecha actual.

	\item \sistema Obtiene de cada noticia clasificada en el paso \ref{CU2:ClasificarN} de la trayectoria principal el \textbf{Título}, \textbf{URL al artículo}, \textbf{Fecha de difusión} y de contar con ello el \textbf{Resumen}.

	\item \sistema Ordena las noticias clasificadas deacuerdo a la regla de negocio \RNref{RN5}{Orden de publicación}.

	\item \sistema \label{CU1:NoticiasR} Muestra la pantalla \Tref{UI3}{Pantalla UI3 Proceso concluido}.

	\item \finCU	

\end{enumerate}



%-------------------------trayectoria Alternativa A-----------------%
\begin{large}
	\Talterna{CU1}{A}\\
\end{large}	
\textbf{Condición:} \textit{El usuario ha presionado el botón cancelar}

\begin{enumerate}[{A-}1.]

	\item \actor Presiona el botón \textbf{Cancelar} de la pantalla \Tref{UI2}{Pantalla UI2 Espera de proceso}.

	\item \sistema Muestra la pantalla \Tref{UI1}{UI1 Inicio}.

	\item \finTA

\end{enumerate}


%-------------------------trayectoria Alternativa B-----------------%
\begin{large}
	\Talterna{CU1}{B}\\
\end{large}	
\textbf{Condición:} \textit{Ya se han recolectado noticias}

\begin{enumerate}[{B-}1.]

	\item \actor Continua en el paso \ref{CU1:NoticiasR} de la trayectoria principal.

	\item \finTA

\end{enumerate}



%================================Puntos de extención=============================%


\begin{large}
	\textbf{Puntos de extensión}\\
\end{large}	

%--------------------Puntos de extención 1------------------------%
\textbf{Causa de la extensión:} El usario desea consultar las noticias clasificadas.\\
\textbf{Región de la trayectorioa:} Proviene del paso \ref{CU1:NoticiasR} de la trayectoria principal.\\
\textbf{Extiende a :} \Tref{CU4}{CU4 Mostrar resultados}\\\\


