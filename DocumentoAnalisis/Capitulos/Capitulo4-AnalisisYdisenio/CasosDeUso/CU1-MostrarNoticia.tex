\Tlabel{CU1}\Tsubsection{CU1 Mostrar noticias}

\begin{large}
	\textbf{Resumen}\\
\end{large}

Brinda al usuario un punto de acceso para visualizar el portal web correspondiente a cada tipo clasificación, ya sea \textbf{Política}, \textbf{Deportes}, \textbf{Ciencia} o \textbf{Economía}; El sitio muestra al menos 15 noticias las cuales se encuentran ordenadas de forma descendente por la fecha de difusión, cada artículo contiene el \textbf{Título} , la \textbf{Fecha de publicación}, si es el caso un \textbf{Resumen de la información}  y un \textbf{Link} el cual direcciona a la página fuente. Cabe mencionar que la primera vez que las publicaciones son mostrados datarán de un periodo anterior a cinco dias de la fecha actual del sistma.\\

\begin{large}
	\textbf{Descripción}\\
\end{large}

\begin{tabular}{|l|l|}
	\hline
	\multicolumn{1}{| >{\columncolor{black}}l|}{ \textcolor{myWhite}{\textbf{Caso de uso: }} }&
	\multicolumn{1}{| >{\columncolor{black}}l|}{ \textcolor{myWhite}{CU1 Mostrar noticias} }\\
	\hline
	\textbf{Actor:} & 	Usuario	\\
	\hline

	\textbf{Propósito:} & Proporcionar una herramienta al usuario para acceder \\
	&a los diferentes tipos de clasificaciones disponibles.\\
	\hline

	\textbf{Entradas:} & Ninguna. \\
	\hline

	\textbf{Salidas:} &$\bullet$ Noticias clasificadas y ordenadas.\\
	&$\bullet$ \textit{Fecha inicio}\\
	&$\bullet$ \textit{Feha fin}\\
	\hline

	\textbf{Precondición:} & El catálogo \textbf{Sitios} debe estar poblado.\\
	\hline

	\textbf{Postcondiciones:} &$\bullet$ El usario tendrá la facultad de consultar las noticias.\\
	&$\bullet$ El usario tendrá la facultad de acceder a los sitios web\\
	&de origen de las noticias.\\
	&$\bullet$ El usario tendrá la facultad consultar noticias de un\\
	&periodo diferente al establecido.\\
	\hline

	\textbf{Reglas de negocio:} &$\bullet$ \RNref{RN2}{Lenguaje de direcciones web} \\
	&$\bullet$ \RNref{RN10}{Fecha de partida}\\
	&$\bullet$ \RNref{RN11}{Orden de publicación}\\
	\hline

	\textbf{Errores:} & $\bullet$ \Tref{MSG1}{MSG1 Catálago vacio}: Se muestra cuando el\\
	&catálogo \textbf{Sitios} No contiene información. \\
	&$\bullet$ \Tref{MSG2}{MSG2 Lenguaje de sitio}: Se muestra cuando los sitios \\
	&proporcionados no se encuentran redactados en lenguaje\\
	& español.\\
	\hline

	\textbf{Autor:} & Carlos Andres Hernandez Gomez \\
	\hline
\end{tabular}\\\\

%--------------------Trayectoria Principal-----------%


\begin{large}
	\textbf{Trayectoria principal}\\
\end{large}	

\begin{enumerate}[1.]
	
	\item \actor Selecciona una opción de la pantalla \Tref{UI1}{UI1 Inicio}, ya sea \textbf{Política}, \textbf{Economía}, \textbf{Deportes} o \textbf{Ciencia}. 
	
	\item \sistema Obtiene el catálogo \textbf{Sitios}.
	
	\item \sistema Verifica que el catálogo \textbf{Sitios} contenga información. \TAref{A}
	
	\item \sistema Verifica que al menos un sitio cumpla con la regla de negocio \RNref{RN2}{Lenguaje de direcciones web}. \TAref{B}
	
	\item \sistema Incluye el caso de uso \Tref{CU2}{CU2 Recolectar noticas}.
	
	\item \sistema \label{CU1:ClasificarN}Incluye el caso de uso \Tref{CU3}{CU3 Calasificar noticias}.
	
	\item \sistema Ordena de forma descendente las noticias clasificadas del paso \ref{CU1:ClasificarN} deacuerdo a su fecha de difusión.
	
	\item \sistema Obtiene la fecha actual.
	
	\item \sistema \label{CU1:FechaI}Calcula el campo \textbf{Fecha incio} deacuerdo a la regla de negocio \RNref{RN10}{Fecha de partida}
	
	\item \sistema Habilita el campo \textbf{Fecha inicio}, \textbf{Fecha fin} y el botón \textbf{Buscar}.
	
	\item \sistema Muestra 15 noticias deacuerdo a la regla de negocio \RNref{RN11}{Orden de publicación}, llena el campo \textbf{Fecha inicio} calculado en el paso \ref{CU1:FechaI} de la trayectoria principal y el campo \textbf{Fecha fin} con la fecha actual, como se visualiza en la pantalla \Tref{UI2}{UI2 Sección política}.

	\item \actor \label{CU1:Consulta}Consulta la información.


	
	\item \finCU	
\end{enumerate}



%--------------------trayectoria Alternativa A---------%
\begin{large}
	\Talterna{A}\\
\end{large}	
\textbf{Condición:} \textit{El catálogo \textbf{Sitios} no contiene información}

\begin{enumerate}[{A-}1.]

	\item \sistema Muestra el mensaje \Tref{MSG1}{MSG1 Catálago vacio} en la pantalla \Tref{UI1}{UI1 Inicio}.
	\item \finCU	

\end{enumerate}

%--------------------trayectoria Alternativa B---------%
\begin{large}
	\Talterna{B}\\
\end{large}	
\textbf{Condición:} \textit{Los sitios proporcionados no estan redactados en lenguaje español}

\begin{enumerate}[{B-}1.]

	\item \sistema Muestra el mensaje  \Tref{MSG2}{Lenguaje de sitio} en la pantalla \Tref{UI1}{UI1 Inicio}.
	\item \finCU	

\end{enumerate}

%--------------------Puntos de extención--------------------%

\begin{large}
	\textbf{Puntos de extensión}\\
\end{large}	

\textbf{Causa de la extensión:} El actor desea consultar noticas en un periodo diferente al establecido.\\
\textbf{Región de la trayectorioa:} Paso \ref{CU1:Consulta} de la trayectoria principal.\\
\textbf{Extiende a :} \Tref{CU6}{CU6 Filtrar noticias por fecha}.\\\\



