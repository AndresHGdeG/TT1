f\Tsubsection{CU1 Mostrar noticias}

\begin{large}
	\textbf{Resumen}\\
\end{large}

Brinda al usuario un punto de acceso para visualizar el portal web correspondiente a cada tipo clasificación, ya sea \textbf{Política}, \textbf{Deportes}, \textbf{Ciencia} o \textbf{Economía}; El sitio muestra al menos 15 noticias las cuales se encuentran ordenadas de forma descendente por la fecha de difusión, cada artículo contiene el \textbf{Título} , la \textbf{Fecha de publicación}, si es el caso un \textbf{Resumen de la información}  y un \textbf{Link} el cual direcciona a la página fuente. Cabe mencionar que los datos mostrados datan des de un mes anterior a la fecha en el cual el actor ha ingresado a la página.\\

\begin{large}
	\textbf{Descripción}
\end{large}

\begin{tabular}{|l|l|}
	\hline
	\multicolumn{1}{| >{\columncolor{black}}l|}{ \textcolor{myWhite}{\textbf{Caso de uso: }} }&
	\multicolumn{1}{| >{\columncolor{black}}l|}{ \textcolor{myWhite}{CU1 Mostrar noticias} }\\
	\hline
	\textbf{Actor:} & 	Usuario	\\
	\hline
	\textbf{Propósito:} & Proporciona una herramienta al usuario para acceder \\
	&a los diferentes tipos de clasificaciones disponibles.\\
	\hline
	\textbf{Entradas:} & Ninguna. \\
	\hline
	\textbf{Salidas:} & Noticias clasificadas y ordenadas.\\
	\hline
	\textbf{Precondición:} & El catálogo \textbf{Direcciónes de sitios} debe estar poblado.\\
	\hline
	\textbf{Postcondiciones:} & El usario tendrá la facultad de consultar las noticias.\\
	\hline
	\textbf{Reglas de negocio:} & {RN2}{Lenguaje de direcciones web}\\
	\hline
	\textbf{Errores:} & Lorem Ipsum \\
	\hline
	\textbf{Autor:} & Lorem Ipsum \\
	\hline
\end{tabular}\\\\

%--------------------Trayectoria Principal-----------%


\begin{large}
	\textbf{Trayectoria principal}
\end{large}	

\begin{enumerate}[1.]
	\item \actor lorem ipsum
	\item \sistema lorem ipsum
	\item \sistema lorem ipsum
	\item \sistema lorem ipsum
	\item \finCU	
\end{enumerate}


%--------------------trayectoria Alternatia A---------%

\begin{large}
	\textbf{Trayectoria alternativa A:}\\
\end{large}	
\textbf{Condición:} \textit{Se escribe la condición}
\begin{enumerate}[{A-}1.]

	\item \actor lorem ipsum
	\item \sistema lorem ipsum
	\item \finTA	

\end{enumerate}


%--------------------trayectoria Alternatia b---------%
\begin{large}
	\textbf{Trayectoria alternativa B:}\\
\end{large}	
\textbf{Condición:} \textit{Se escribe la condición}

\begin{enumerate}[{B-}1.]

	\item \actor lorem ipsum
	\item \sistema lorem ipsum
	\item \finTA	

\end{enumerate}


%--------------------Puntos de extención--------------------%

\begin{large}
	\textbf{Puntos de extensión}\\
\end{large}	

\textbf{Causa de la extensión:} Lorem ipsum\\
\textbf{Región de la trayectorioa:} Lorem ipsum\\
\textbf{Extiende a :} Lorem ipsum\\\\

\textbf{Causa de la extensión:} Lorem ipsum\\
\textbf{Región de la trayectorioa:} Lorem ipsum\\
\textbf{Extiende a :} Lorem ipsum\\

