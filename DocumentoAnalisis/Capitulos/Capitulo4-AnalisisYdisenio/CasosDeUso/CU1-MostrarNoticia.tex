\Tlabel{CU1}\Tsubsection{CU1 Mostrar noticias}

\begin{large}
	\textbf{Resumen}\\
\end{large}

Brinda al usuario un punto de acceso para visualizar el portal web correspondiente a cada tipo clasificación, ya sea \textbf{Política}, \textbf{Deportes}, \textbf{Ciencia} o \textbf{Economía}; El sitio muestra al menos 15 noticias las cuales se encuentran ordenadas de forma descendente por la fecha de difusión, cada artículo contiene el \textbf{Título} , la \textbf{Fecha de publicación}, si es el caso un \textbf{Resumen de la información}  y un \textbf{Link} el cual direcciona a la página fuente. Cabe mencionar que la primera vez que los datos son mostrados datan de un mes anterior a la fecha consultada.\\

\begin{large}
	\textbf{Descripción}\\
\end{large}

\begin{tabular}{|l|l|}
	\hline
	\multicolumn{1}{| >{\columncolor{black}}l|}{ \textcolor{myWhite}{\textbf{Caso de uso: }} }&
	\multicolumn{1}{| >{\columncolor{black}}l|}{ \textcolor{myWhite}{CU1 Mostrar noticias} }\\
	\hline
	\textbf{Actor:} & 	Usuario	\\
	\hline
	\textbf{Propósito:} & Proporcionar una herramienta al usuario para acceder \\
	&a los diferentes tipos de clasificaciones disponibles.\\
	\hline
	\textbf{Entradas:} & Ninguna. \\
	\hline
	\textbf{Salidas:} & Noticias clasificadas y ordenadas.\\
	\hline
	\textbf{Precondición:} & El catálogo \textbf{Sitios} debe estar poblado.\\
	\hline
	\textbf{Postcondiciones:} & El usario tendrá la facultad de consultar las noticias.\\
	\hline
	\textbf{Reglas de negocio:} &\RNref{RN2}{Lenguaje de direcciones web} \\
	\hline
	\textbf{Errores:} & \Tref{MSG1}{MSG1 Catálago vacio}: Se muestra cuando el catálogo \textbf{Sitios} No contiene\\
	& información. \\
	\hline
	\textbf{Autor:} & Carlos Andres Hernandez Gomez \\
	\hline
\end{tabular}\\\\

%--------------------Trayectoria Principal-----------%


\begin{large}
	\textbf{Trayectoria principal}\\
\end{large}	

\begin{enumerate}[1.]
	\item \actor Selecciona la opción \textbf{Política} de la pantalla \Tref{UI1}{UI1 Inicio}. \TAref{A} \TAref{B} \TAref{C}
	\item \sistema Obtiene el catálogo \textbf{Sitios}.
	\item \sistema Verifica que el catálogo \textbf{Sitios} contenga información. \TAref{D}
	\item \sistema 
	\item \finCU	
\end{enumerate}


%--------------------trayectoria Alternativa A---------%

\begin{large}
	\Talterna{A}\\
\end{large}	
\textbf{Condición:} \textit{Selecciona la opción \textbf{Deportes}}
\begin{enumerate}[{A-}1.]
	\item \actor lorem ipsum
	\item \sistema lorem ipsum
	\item \finTA	

\end{enumerate}


%--------------------trayectoria Alternativa B---------%
\begin{large}
	\Talterna{B}\\
\end{large}	
\textbf{Condición:} \textit{Selecciona la opción \textbf{Ciencia}}

\begin{enumerate}[{B-}1.]

	\item \actor lorem ipsum
	\item \sistema lorem ipsum
	\item \finTA	

\end{enumerate}


%--------------------trayectoria Alternativa C---------%
\begin{large}
	\Talterna{C}\\
\end{large}	
\textbf{Condición:} \textit{Selecciona la opción \textbf{Economía}}

\begin{enumerate}[{C-}1.]

	\item \actor lorem ipsum
	\item \sistema lorem ipsum
	\item \finTA	

\end{enumerate}

%--------------------trayectoria Alternativa D---------%
\begin{large}
	\Talterna{D}\\
\end{large}	
\textbf{Condición:} \textit{El catálogo \textbf{Sitios} no contiene información}

\begin{enumerate}[{D-}1.]

	\item \sistema Muestra el mensaje \Tref{MSG1}{MSG1 Catálago vacio} en la pantalla \Tref{UI1}{UI1 Inicio}.
	\item \finCU	

\end{enumerate}

%--------------------Puntos de extención--------------------%

\begin{large}
	\textbf{Puntos de extensión}\\
\end{large}	

\textbf{Causa de la extensión:} Lorem ipsum\\
\textbf{Región de la trayectorioa:} Lorem ipsum\\
\textbf{Extiende a :} Lorem ipsum\\\\

\textbf{Causa de la extensión:} Lorem ipsum\\
\textbf{Región de la trayectorioa:} Lorem ipsum\\
\textbf{Extiende a :} Lorem ipsum\\

