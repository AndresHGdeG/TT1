\Tlabel{CU1}\Tsubsection{CU1 Seleccionar sección}

\begin{large}
	\textbf{Resumen}\\
\end{large}

Brinda al usuario un punto de acceso para elegir una sección; Las clasificaciones son, \textbf{Política}, \textbf{Deportes}, \textbf{Ciencia} y \textbf{Economía} en cada una se podrá consultar noticias, artículos y publicaciones dentro de un intervalo de tiempo específico; La fuente de información será un diccionario de los sitios web mas consultados y confiables en México. Cabe mencionar que el intervalo de tiempo por defecto es anterior en cinco dias de la fecha que se ha ingresa al portal.\\

\begin{large}
	\textbf{Descripción}\\
\end{large}

%============================================Tabla 1======================================%

\begin{tabular}{|l|l|}

%-----------------------Ecanbezado--------------------------%
	\hline
	\multicolumn{1}{| >{\columncolor{black}}l|}{ \textcolor{myWhite}{\textbf{Caso de uso: }} }&
	\multicolumn{1}{| >{\columncolor{black}}l|}{ \textcolor{myWhite}{CU1 Seleccionar sección} }\\
	\hline

%-----------------------Actor--------------------------------%
	\textbf{Actor:} & 	Usuario	\\
	\hline

%-----------------------Propósito------------------------------%
	\textbf{Propósito:} & Proporcionar una herramienta para acceder \\
	&a los diferentes tipos de clasificaciones disponibles.\\
	\hline

%----------------------Entradas--------------------------------%
	\textbf{Entradas:} & Ninguna. \\
	\hline

%-----------------------Salidas---------------------------------%
	\textbf{Salidas:} &$\bullet$ \textit{Fecha inicio}\\
	&$\bullet$ \textit{Feha fin}\\
	\hline

%-----------------------Precondiciones------------------------------%
	\textbf{Precondición:} & El catálogo \textbf{Direcciones web} debe estar poblado.\\
	\hline

%-----------------------Postcondiciones-----------------------------%

%---------Post 1----------%
	\textbf{Postcondiciones:} &$\bullet$ El usario tendrá la facultad de buscar noticias\\
	&\ \ de la sección elegida\\
%---------Post 2----------%
	&$\bullet$ El usario tendrá la facultad de establecer un \\
	&\ \ intervalo de tiempo para buscar los artículos\\
	\hline

%-----------------------Reglas de negocio----------------------------%
	\textbf{Reglas de negocio:} &$\bullet$ \RNref{RN2}{Lenguaje de direcciones web} \\
	&$\bullet$ \RNref{RN7}{Perido preestablecido}\\
	\hline

%---------------------------Errores-----------------------------------%

%------Error 1----------%
	\textbf{Errores:} & $\bullet$ \TError{CU1}{Uno} Cuando el  catálogo \textbf{Direcciones web}\\
	&\ \ no contiene información se muestra el mensaje\\
	&\ \  \Tref{MSG1}{MSG1 Catálago vacio}, fin del caso de uso\\
%------Error 2----------%
	&$\bullet$ \TError{CU1}{Dos} Cuando los sitios proporcionados no se\\
	&\ \ \ encuentran redactados en lenguaje  español se\\
	&\ \ muestra el mensaje \Tref{MSG2}{MSG2 Lenguaje de sitio}, fin\\
	&\ \ del caso de uso\\
	\hline

%-------------------------Autor---------------------------------------%
	\textbf{Autor:} & Carlos Andres Hernandez Gomez \\
	\hline
\end{tabular}\\\\

%--------------------Trayectoria Principal-----------%


\begin{large}
	\textbf{Trayectoria principal}\\
\end{large}	

\begin{enumerate}[1.]
	
	\item \actor Selecciona una opción de la pantalla \Tref{UI1}{UI1 Inicio}; \textbf{Política}, \textbf{Economía}, \textbf{Deportes} o \textbf{Ciencia}. 
	
	\item \sistema Obtiene el catálogo \textbf{Direcciones web}.
	
	\item \sistema Verifica que el catálogo \textbf{Direcciones web} contenga información. \TEref{CU1}{Uno}
	
	\item \sistema Verifica que al menos un sitio cumpla con la regla de negocio \RNref{RN2}{Lenguaje de direcciones web}. \TEref{CU1}{Dos}

	\item \sistema Obtiene la fecha actual.

	\item \sistema \label{CU1:FechaI}Calcula el campo \textbf{Fecha incio} y \textbf{Fecha fin} deacuerdo a la regla de negocio \RNref{RN7}{Perido preestablecido}.

	\item \label{CU1:Pantalla}\sistema Muestra deshabilitado y con fechas los campos \textbf{Fecha inicio} y \textbf{Fecha fin} con lo antes calculado, como se visualiza en la pantalla \Tref{UI2}{UI2 Sección política}.
	
	\item \finCU	

\end{enumerate}






\textbf{Causa de la extensión:} El actor desea consultar las noticias de una sección.\\
\textbf{Región de la trayectorioa:} Paso \ref{CU1:Pantalla} de la trayectoria principal.\\
\textbf{Extiende a :} \Tref{CU2}{CU2 Buscar noticia}.\\\\



