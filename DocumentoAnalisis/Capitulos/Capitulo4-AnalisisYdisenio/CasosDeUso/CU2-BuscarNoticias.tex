\Tlabel{CU2}\Tsubsection{CU2 Buscar noticias}

%================================Intriodcucción==================================%
%----------------------Resumen-----------------------------------%

\begin{large}
	\textbf{Resumen}\\
\end{large}

Permite al actor realizar una busqueda de noticias correspondiente a la sección elegida, ya sea \textbf{Política}, \textbf{Deportes}, \textbf{Ciencia} o \textbf{Economía}; La consulta se realiza en un periodo preestablecido el cual data 5 días antes a al fecha actual o se ingresar un intervalo de tiempo diferente. El sitio muestra 15 noticias ordenadas de forma descendente deacuerdo a la fecha de difusión; Cada artículo contiene el \textbf{Título} , la \textbf{Fecha de publicación},un \textbf{Link} el cual direcciona a la página fuente y de contar con ello  un \textbf{Resumen de la información}.\\

\begin{large}
	\textbf{Descripción}\\
\end{large}



%=====================================Tabla 1====================================%

\begin{tabular}{|l|l|}

%-----------------------Ecanbezado-----------------------------------%
	\hline
	\multicolumn{1}{| >{\columncolor{black}}l|}{ \textcolor{myWhite}{\textbf{Caso de uso: }} }&
	\multicolumn{1}{| >{\columncolor{black}}l|}{ \textcolor{myWhite}{CU2 Buscar noticias} }\\
	\hline
%-----------------------Actor----------------------------------------%
	\textbf{Actor:} & 	Usuario	\\
	\hline
%-----------------------Propósito------------------------------------%
	\textbf{Propósito:} & Brindar una herramienta que permita recolectar y clasificar\\
	&noticias de una sección específica en un periodo de tiempo.\\
	\hline

%----------------------Entradas--------------------------------------%
	\textbf{Entradas:} &$\bullet$ \textit{Fecha inicio}\\
	&$\bullet$ \textit{Feha fin}\\
	\hline

%-----------------------Salidas--------------------------------------%
	\textbf{Salidas:}&Noticias clasificadas; De cada una se muestra:\\
	&$\bullet$ \textbf{Título}\\
	&$\bullet$ \textbf{Nombre de la página fuente}\\
	&$\bullet$ \textbf{Link al artículo}\\
	&$\bullet$ \textbf{Fecha de difusión}\\
	&$\bullet$ \textbf{Resumen}\\
	\hline

%-----------------------Precondiciones-------------------------------%

	\textbf{Precondición:} & Una sección debe estar seleccionada.\\
	\hline

%-----------------------Postcondiciones------------------------------%

%---------Post 1----------%
	\textbf{Postcondiciones:} &$\bullet$ El usario tendrá la facultad de consultar las noticias\\
%---------Post 2----------%
	&$\bullet$ El usario tendrá la facultad de acceder a los sitios web\\
	&\ \ de las noticias recolectadas\\
	\hline

%-----------------------Reglas de negocio----------------------------%
	\textbf{Reglas de negocio:}&$\bullet$ \RNref{RN6}{Formato de fecha}\\
	&$\bullet$ \RNref{RN7}{Perido preestablecido}\\
	&$\bullet$ \RNref{RN8}{Periodo válido}\\
	&$\bullet$ \RNref{RN9}{Límite de periodo}\\
	&$\bullet$ \RNref{RN10}{Campos obligatorios}\\
	&$\bullet$ \RNref{RN11}{Orden de publicación}\\
	\hline
%---------------------------Errores----------------------------------%

%------Error 1----------%
	\textbf{Errores:} &$\bullet$\TError{CU2}{Uno} Cuando hay campos vacios marcados como obliga-\\
	&\ \ torios se muestra el mensaje \Tref{MSG3}{MSG3 Faltan campos obliga-}\\
	&\ \ \Tref{MSG3}{torios} y continua en el paso \ref{CU2:FechaI} de la trayectoria alternativa\\
	&\ \  A\\
%------Error 2----------%
	&$\bullet$\TError{CU2}{Dos} Cuando el usario ha ingresado una fecha no válido\\
	&\ \ se muestra el mensaje \Tref{MSG4}{MSG4 Formato de fecha inválido}\\
	&\ \ y continua en el paso  \ref{CU2:FechaI} de la trayectoria alternativa A\\
	\hline
\end{tabular}\\\\

%============================================Tabla 2======================================%
\begin{tabular}{|l|l|}

%-----------------------Ecanbezado-----------------------------------%
	\hline
	\multicolumn{1}{| >{\columncolor{black}}l|}{ \textcolor{myWhite}{\textbf{Caso de uso: }} }&
	\multicolumn{1}{| >{\columncolor{black}}l|}{ \textcolor{myWhite}{CU2 Buscar noticias} }\\
	\hline


%------Error 3----------%
	\textbf{Errores:}&$\bullet$\TError{CU2}{Tres} Cuando existe incongruencias en los periodos de la\\
	&\ \ fehca se muesntra el mensaje \Tref{MSG5}{MSG5 Periodo no válido} y\\
	&\ \ continua en el paso  \ref{CU2:FechaI} de la trayectoria alternativa A \\
%-----Error 4----------%
	&$\bullet$\TError{CU3}{Cuatro} Cuando el usario ha introducido una intervalo\\
	&\ \ de tiempo fuera de los límites se muesta el mensaje \\
	&\ \ \Tref{MSG6}{MSG6 Litimes fuera de rango} y continua en el paso \ref{CU2:FechaI} de\\
	&\ \ la trayectoria alternativa A\\
	\hline 
%-------------------------Autor--------------------------------------%

	\textbf{Autor:} & Carlos Andres Hernandez Gomez \\
	\hline
\end{tabular}\\\\

%============================Trayectorias========================================%

%-----------------------Trayectoria Principal-----------------------%

\begin{large}
	\textbf{Trayectoria principal}\\
\end{large}	

\begin{enumerate}[1.]

	\item \actor \label{CU2:Buscar}Da click en el botón \textbf{Buscar} de la pantalla \Tref{UI2}{UI2 Sección política}. \TAref{CU2}{A} \TAref{CU2}{B}

	\item \sistema \label{CU2:CamposO}Verifica la regla de negocio \RNref{RN10}{Campos obligatorios}. \TEref{CU2}{Uno}

	\item \sistema Verifica la regla de negocio \RNref{RN6}{Formato de fecha}. \TEref{CU2}{Dos}

	\item \sistema Verifica la regla de negocio \RNref{RN8}{Periodo válido}. \TEref{CU2}{Tres}

	\item \sistema Verifica la regla de negocio \RNref{RN9}{Límite de periodo}. \TEref{CU2}{Cuatro} 

	\item \sistema Incluye el caso de uso \Tref{CU3}{CU3 Recolectar noticas}.

	\item \sistema \label{CU2:ClasificarN}Incluye el caso de uso \Tref{CU4}{CU4 Calasificar noticias}.

	\item \sistema \label{CU2:DatosN}Obtiene de cada noticia clasificada en el paso \ref{CU2:ClasificarN} de la trayectoria principal el \textbf{Título}, \textbf{Nombre de la página fuente}, \textbf{Link al artículo}, \textbf{Fecha de difusión} y de contar con ello el \textbf{Resumen}.

	\item \sistema \label{CU2:OrdenaN}Ordena de forma descendente las noticias clasificadas del paso \ref{CU2:ClasificarN} de la trayectoria principal deacuerdo a su fecha de difusión.

	\item \sistema Muestra 15 noticias de las ordenadas, deacuerdo a la regla de negocio \RNref{RN12}{Orden de publicación} con la información obtenida en el paso \ref{CU2:DatosN} de la trayectoria principal, como se visualiza en la pantalla \Tref{UI3}{UI3 Resultados de búsqueda} 

	\item \actor \label{CU2:Consulta}Consulta la información.

	\item \finCU	
\end{enumerate}

\newpage
%-------------------------trayectoria Alternativa A-----------------%
\begin{large}
	\Talterna{CU2}{A}\\
\end{large}	
\textbf{Condición:} \textit{El botón \textbf{Cambiar periodo} es presionado cuando está en estado \textit{Off}}

\begin{enumerate}[{A-}1.]

	\item \sistema Habilita y limpia el campo \textbf{Fecha inicio} y \textbf{Fecha fin}, como se muestra
	en la pantalla \Tref{UI2_1}{UI2.1 Cambio de periodo}.

	\item \actor \label{CU2:FechaI}Ingresa el campo \textbf{Fecha inicio}.

	\item \actor Ingresa el campo \textbf{Fecha fin}.

	\item \actor Da click en el botón \textbf{Buscar} de la pantalla \Tref{UI2_1}{UI2.1 Cambio de periodo}.

	\item \sistema Continua en el paso \ref{CU2:CamposO} de la trayectoria principal.

	\item \finTA

\end{enumerate}


%-------------------------trayectoria Alternativa B-----------------%
\begin{large}
	\Talterna{CU2}{B}\\
\end{large}	
\textbf{Condición:} \textit{El botón \textbf{Cambiar periodo} es presionado cuando está en estado \textit{On}}

\begin{enumerate}[{B-}1.]

	\item \sistema Obtiene la fecha actual.
	
	\item \sistema Calcula el campo \textbf{Fecha incio} y \textbf{Fecha fin} deacuerdo a la regla de negocio \RNref{RN7}{Perido preestablecido}.

	\item \label{CU1:Pantalla}\sistema Muestra deshabilitado y con fechas los campos \textbf{Fecha inicio} y \textbf{Fecha fin} con lo antes calculado, como se visualiza en la pantalla \Tref{UI2}{UI2 Sección política}.

	\item \sistema Continua en el paso \ref{CU2:Buscar} de la trayectoria principal.

	\item \finTA

\end{enumerate}

%================================Puntos de extención=============================%


\begin{large}
	\textbf{Puntos de extensión}\\
\end{large}	

%--------------------Puntos de extención 1------------------------%
\textbf{Causa de la extensión:} Lorem ipsum\\
\textbf{Región de la trayectorioa:} Lorem ipsum\\
\textbf{Extiende a :} Lorem ipsum\\\\