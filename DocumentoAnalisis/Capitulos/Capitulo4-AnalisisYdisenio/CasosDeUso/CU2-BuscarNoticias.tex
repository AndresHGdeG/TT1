\Tlabel{CU2}\Tsubsection{CU2 Buscar noticias}

\begin{large}
	\textbf{Resumen}\\
\end{large}

Permite al actor realizar una busqueda de las noticias correspondiente a cada tipo clasificación, ya sea \textbf{Política}, \textbf{Deportes}, \textbf{Ciencia} o \textbf{Economía}; El sitio muestra al menos 15 noticias las cuales se encuentran ordenadas de forma descendente por la fecha de difusión; Cada artículo contiene el \textbf{Título} , la \textbf{Fecha de publicación}, si es el caso un \textbf{Resumen de la información}  y un \textbf{Link} el cual direcciona a la página fuente.\\

\begin{large}
\textbf{Descripción}\\
\end{large}

\begin{tabular}{|l|l|}
\hline
\multicolumn{1}{| >{\columncolor{black}}l|}{ \textcolor{myWhite}{\textbf{Caso de uso: }} }&
\multicolumn{1}{| >{\columncolor{black}}l|}{ \textcolor{myWhite}{CU2 Buscar noticias} }\\
\hline
\textbf{Actor:} & 	Usuario	\\
\hline

\textbf{Propósito:} & Proporcionar una herramienta para  \\
&recolectar y clasificar noticias de una sección específica.\\
\hline

\textbf{Entradas:} &$\bullet$ \textit{Fecha inicio}\\
&$\bullet$ \textit{Feha fin}\\
\hline

\textbf{Salidas:}&Noticias clasificadas; De cada una se muestra:\\
&\ \ $\bullet$ \textbf{Título}\\
&\ \ $\bullet$ \textbf{Nombre de la página fuente}\\
&\ \ $\bullet$ \textbf{Link al artículo}\\
&\ \ $\bullet$ \textbf{Fecha de difusión}\\
&\ \ $\bullet$ \textbf{Resumen}\\
\hline

\textbf{Precondición:} & El campo \textbf{Fecha inicio} y \textbf{Fecha fin} deben contener\\
&información.\\
\hline

\textbf{Postcondiciones:} &$\bullet$ El usario tendrá la facultad de consultar las noticias\\
&$\bullet$ El usario tendrá la facultad de acceder a los sitios web\\
&\ \ de las noticias recolectadas\\
\hline

\textbf{Reglas de negocio:} &$\bullet$ \RNref{RN11}{Orden de publicación}\\
\hline

\textbf{Errores:} Ninguno.\\
\hline

\textbf{Autor:} & Carlos Andres Hernandez Gomez \\
\hline
\end{tabular}\\\\

%%--------------------Trayectoria Principal-----------
\begin{large}
\textbf{Trayectoria principal}\\
\end{large}	

\begin{enumerate}[1.]

\item \actor Da click en el botón \textbf{Buscar} de la pantalla \Tref{UI2}{UI2 Sección política}. \TAref{A}

\item \sistema Incluye el caso de uso \Tref{CU3}{CU3 Recolectar noticas}.

\item \sistema \label{CU2:ClasificarN}Incluye el caso de uso \Tref{CU4}{CU4 Calasificar noticias}.

\item \sistema \label{CU2:DatosN}Obtiene de cada noticia clasificada en el paso \ref{CU2:ClasificarN} de la trayectoria principal el \textbf{Título}, \textbf{Nombre de la página fuente}, \textbf{Link al artículo}, \textbf{Fecha de difusión} y de contar con ello el \textbf{Resumen}.

\item \sistema \label{CU2:OrdenaN}Ordena de forma descendente las noticias clasificadas del paso \ref{CU2:ClasificarN} de la trayectoria principal deacuerdo a su fecha de difusión.

\item \sistema Muestra 15 noticias de las ordenadas, deacuerdo a la regla de negocio \RNref{RN11}{Orden de publicación} con la información obtenida en el paso \ref{CU2:DatosN} de la trayectoria principal, como se visualiza en la pantalla \Tref{UI3}{UI3 Resultados de búsqueda} 

\item \actor \label{CU2:Consulta}Consulta la información.

\item \finCU	
\end{enumerate}


%%--------------------trayectoria Alternativa A----------
\begin{large}
\Talterna{A}\\
\end{large}	
\textbf{Condición:} \textit{El actor ha presionado el botón \textbf{Cambiar periodo}}

\begin{enumerate}[{A-}1.]

\item \sistema 
\item \finCU	

\end{enumerate}



