\Tlabel{CU6}\Tsubsection{CU6 Filtrar noticias por fecha}

\begin{large}
	\textbf{Resumen}\\
\end{large}

Permite al usuario cambiar el periodo establecido para la recolección y clasificación de noticias; Se realiza deacuerdo a la sección seleccionada como \textbf{Política}, \textbf{Ciencia}, \textbf{Deportes} o \textbf{Economía}.\\

\begin{large}
	\textbf{Descripción}\\
\end{large}

\begin{tabular}{|l|l|}
	\hline
	\multicolumn{1}{| >{\columncolor{black}}l|}{ \textcolor{myWhite}{\textbf{Caso de uso: }} }&
	\multicolumn{1}{| >{\columncolor{black}}l|}{ \textcolor{myWhite}{CU6 Filtrar noticias por fecha} }\\
	\hline

	\textbf{Actor:} & Usuario	\\
	\hline

	\textbf{Propósito:} & Brindar una herramienta al usuario para establecer un\\
	&filtro deacuerdo a la fechas de difusión de los artículos.\\
	\hline

	\textbf{Entradas:} &$\bullet$ \textbf{Cambiar periodo:} Se selecciona mediante el mouse\\
	&$\bullet$ \textbf{Fecha de inicio:} Se ingresa con el teclado\\
	&$\bullet$ \textbf{Fecha de fin:} Se ingresa con el teclado\\
	\hline

	\textbf{Salidas:} & Lorem Ipsum\\
	\hline

	\textbf{Precondición:} &El usario ha seleccionado una sección como \textbf{Política},\\
	& \textbf{Ciencia}, \textbf{Deportes} o \textbf{Economía}\\
	\hline

	\textbf{Postcondiciones:}&$\bullet$ El usario tendrá la facultad de consultar las noticias.\\
	&$\bullet$ El usario tendrá la facultad de acceder a los sitios web\\
	&\ \ de las noticias recolectadas\\
	&$\bullet$ El usario tendrá la facultad de consultar noticias de un\\
	&\ \ periodo diferente al establecido\\
	\hline

	\textbf{Reglas de negocio:} & Lorem Ipsum \\
	\hline

	\textbf{Errores:} & Lorem Ipsum \\
	\hline

	\textbf{Autor:} & Lorem Ipsum \\
	\hline

\end{tabular}\\\\

%--------------------Trayectoria Principal-----------%


\begin{large}
	\textbf{Trayectoria principal}\\
\end{large}	

\begin{enumerate}[1.]
	\item \actor lorem ipsum
	\item \sistema lorem ipsum
	\item \sistema lorem ipsum
	\item \sistema lorem ipsum
	\item \finCU	
\end{enumerate}


%--------------------trayectoria Alternatia A---------%

\begin{large}
	\textbf{Trayectoria alternativa A:}\\
\end{large}	
\textbf{Condición:} \textit{Se escribe la condición}
\begin{enumerate}[{A-}1.]

	\item \actor lorem ipsum
	\item \sistema lorem ipsum
	\item \finTA	

\end{enumerate}


%--------------------trayectoria Alternatia b---------%
\begin{large}
	\textbf{Trayectoria alternativa B:}\\
\end{large}	
\textbf{Condición:} \textit{Se escribe la condición}

\begin{enumerate}[{B-}1.]

	\item \actor lorem ipsum
	\item \sistema lorem ipsum
	\item \finTA	

\end{enumerate}


%--------------------Puntos de extención--------------------%

\begin{large}
	\textbf{Puntos de extensión}\\
\end{large}	

\textbf{Causa de la extensión:} Lorem ipsum\\
\textbf{Región de la trayectorioa:} Lorem ipsum\\
\textbf{Extiende a :} Lorem ipsum\\\\

\textbf{Causa de la extensión:} Lorem ipsum\\
\textbf{Región de la trayectorioa:} Lorem ipsum\\
\textbf{Extiende a :} Lorem ipsum\\

