\Tlabel{CU4}\Tsubsection{CU4 Mostrar resultados}

%================================Intriodcucción==================================%
%----------------------Resumen-----------------------------------%

\begin{large}
	\textbf{Resumen}\\
\end{large}

Permite al actor vizualizar las noticias correspondiente a la sección elegida,
ya sea \textbf{Política}, \textbf{Deportes}, \textbf{Ciencia y técnología}, \textbf{Economía}
o \textbf{Cultura}. La consulta se realiza en un periodo establecido; El sitio muestra 15 
noticias ordenadas de forma descendente deacuerdo a la fecha de publicación, cada artículo 
contiene el \textbf{Título} , la \textbf{Fecha de publicación}, \textbf{URL} el cual direcciona a la página fuente que ha proporcionado la noticias y de contar con ello  un
\textbf{Resumen de la información}.\\

\begin{large}
	\textbf{Descripción}\\
\end{large}



%=====================================Tabla 1====================================%

\begin{tabular}{|l|l|}

%-----------------------Ecanbezado-----------------------------------%
	\hline
	\multicolumn{1}{| >{\columncolor{black}}l|}{ \textcolor{myWhite}{\textbf{Caso de uso: }} }&
	\multicolumn{1}{| >{\columncolor{black}}l|}{ \textcolor{myWhite}{CU2 Buscar noticias} }\\
	\hline
%-----------------------Actor----------------------------------------%
	\textbf{Actor:} & 	Usuario.	\\
	\hline
%-----------------------Propósito------------------------------------%
	\textbf{Propósito:} & Brindar una herramienta que permita consultar\\
	&las noticias clasificadas\\
	\hline

%----------------------Entradas--------------------------------------%
	\textbf{Entradas:} &Ninguna.\\
	\hline

%-----------------------Salidas--------------------------------------%
	\textbf{Salidas:}&Noticias clasificadas; De cada una se muestra:\\
	&$\bullet$ \textbf{Título}\\
	&$\bullet$ \textbf{URL al artículo}\\
	&$\bullet$ \textbf{Fecha de publicación}\\
	&$\bullet$ \textbf{Resumen}\\
	\hline

%-----------------------Precondiciones-------------------------------%

	\textbf{Precondición:} & La clasificación de las noticias debe estar\\
	&completa\\
	\hline

%-----------------------Postcondiciones------------------------------%

%---------Post 1----------%
	\textbf{Postcondiciones:}& Ninguna. \\
	\hline

%-----------------------Reglas de negocio----------------------------%
	\textbf{Reglas de negocio:}& \RNref{RN5}{Orden de publicación}\\
	\hline
%---------------------------Errores----------------------------------%

%------Error 1----------%
	\textbf{Errores:} &Ninguno.\\
	\hline
%-------------------------Autor--------------------------------------%

	\textbf{Autor:} & Carlos Andres Hernandez Gomez \\
	\hline
\end{tabular}\\\\

%============================Trayectorias========================================%

%-----------------------Trayectoria Principal-----------------------%

\begin{large}
	\textbf{Trayectoria principal}\\
\end{large}	

\begin{enumerate}[1.]

	\item \actor Presiona el botón 	\textbf{Aceptar} de la pantalla \Tref{UI3}{Pantalla UI3 Proceso concluido}. \TAref{CU1}{A}

	\item \sistema \label{CU2:MostrarN}Muestra 15 noticias de las ordenadas previamente, deacuerdo a la regla de negocio \RNref{RN5}{Orden de publicación} las cuales cumplan con el filtro de sección y fecha,  como se visualiza en la pantalla \Tref{UI2}{UI2 Sección política} 

	\item \actor \label{CU1:Consulta}Consulta la información.\TAref{CU2}{B}

	\item \finCU	
\end{enumerate}


%-------------------------trayectoria Alternativa A-----------------%
\begin{large}
	\Talterna{CU2}{A}\\
\end{large}	
\textbf{Condición:} \textit{El usuario ha presionado el botón cancelar}

\begin{enumerate}[{A-}1.]

	\item \actor Presiona el botón \textbf{Cancelar} de la pantalla \Tref{UI3}{Pantalla UI3 Proceso concluido}.

	\item \sistema Muestra la pantalla \Tref{UI1}{UI1 Inicio}.

	\item \finTA

\end{enumerate}


%-------------------------trayectoria Alternativa B-----------------%
\begin{large}
	\Talterna{CU2}{B}\\
\end{large}	
\textbf{Condición:} \textit{El usuario ha cambiado el periodo establecido}

\begin{enumerate}[{B-}1.]

	\item \actor Presiona un botón del menú \textbf{Cambio de periodo} de la pantalla \Tref{UI2}{UI2 Sección política}.

	\item \sistema Continua en el paso \ref{CU2:MostrarN} de la trayectoria principal con el periodo seleccionado.

	\item \finTA

\end{enumerate}

