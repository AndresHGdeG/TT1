\Tlabel{CU4}\Tsubsection{CU4 Mostrar resultados}

%================================Intriodcucción==================================%
%----------------------Resumen-----------------------------------%

\begin{large}
	\textbf{Resumen}\\
\end{large}

Permite al actor visualizar las noticias correspondiente a la sección elegida,
ya sea \textbf{Política}, \textbf{Deportes}, \textbf{Ciencia y tecnología}, \textbf{Economía}
o \textbf{Cultura}. La consulta se realiza en un periodo establecido; el sitio muestra hasta 15
 noticias, cada artículo contiene el \textbf{Título} , la \textbf{Fecha de publicación}, \textbf{URL} el cual direcciona a la página fuente que ha proporcionado la noticias y de contar con ello  un
\textbf{Resumen de la información}.\\

\begin{large}
	\textbf{Descripción}\\
\end{large}



%=====================================Tabla 1====================================%

\begin{tabular}{|l|l|}

%-----------------------Ecanbezado-----------------------------------%
	\hline
	\multicolumn{1}{| >{\columncolor{black}}l|}{ \textcolor{myWhite}{\textbf{Caso de uso: }} }&
	\multicolumn{1}{| >{\columncolor{black}}l|}{ \textcolor{myWhite}{CU4 Mostrar resultados} }\\
	\hline
%-----------------------Actor----------------------------------------%
	\textbf{Actor:} & 	Usuario	\\
	\hline
%-----------------------Propósito------------------------------------%
	\textbf{Propósito:} & Brindar una herramienta que permita consultar las\\
	&noticias clasificadas\\
	\hline

%----------------------Entradas--------------------------------------%
	\textbf{Entradas:} &Periodo de búsqueda: Se selecciona con el mouse \\
	\hline

%-----------------------Salidas--------------------------------------%
	\textbf{Salidas:}&$\bullet$ \Tref{MSG2}{MSG2 Petición vacía}\\
	&$\bullet$ Noticias clasificadas; de cada una se muestra:\\
	&\ \ $\circ$ \textbf{Título}\\
	&\ \ $\circ$ \textbf{URL al artículo}\\
	&\ \ $\circ$ \textbf{Fecha de publicación}\\
	&\ \ $\circ$ \textbf{Resumen}\\
	\hline

%-----------------------Precondiciones-------------------------------%

	\textbf{Precondición:} & La clasificación de las noticias debe estar completa\\
	\hline

%-----------------------Postcondiciones------------------------------%

%---------Post 1----------%
	\textbf{Postcondiciones:}& Ninguna\\
	\hline

%-----------------------Reglas de negocio----------------------------%
	\textbf{Reglas de negocio:}& \RNref{RN5}{Orden de publicación}\\
	\hline
%---------------------------Errores----------------------------------%

%------Error 1----------%
	\textbf{Errores:} &\TError{CU4}{Uno} Cuando no se ha encontrado noticias en el\\
	&\ \ día seleccionado se muestra el mensaje \Tref{MSG2}{MSG2}\\
	&\ \ \Tref{MSG2}{Petición vacía}\\
	\hline

\end{tabular}\\\\

%============================Trayectorias========================================%

%-----------------------Trayectoria Principal-----------------------%

\begin{large}
	\textbf{Trayectoria principal}\\
\end{large}	

\begin{enumerate}[1.]

	\item \actor Presiona el botón 	\textbf{Aceptar} de la pantalla \Tref{UI3}{Pantalla UI3 Proceso concluido}. \TAref{CU1}{A}


	\item \sistema Obtiene la fecha actual.

	\item \sistema Muestra hasta 15 de acuerdo a la regla de negocio \RNref{RN5}{Orden de publicación} de la sección previamente elegida (filtro de sección) y del día actual (filtro de fecha), como se visualiza en la pantalla \Tref{UI4}{UI4 Resultados de consulta}.

	\item \actor \label{CU4:Consulta}Consulta la información.\TAref{CU4}{B}

	\item \finCU	
\end{enumerate}


%-------------------------trayectoria Alternativa A-----------------%
\begin{large}
	\Talterna{CU4}{A}\\
\end{large}	
\textbf{Condición:} \textit{El usuario ha presionado el botón cancelar}

\begin{enumerate}[{A-}1.]

	\item \actor Presiona el botón \textbf{Cancelar} de la pantalla \Tref{UI3}{Pantalla UI3 Proceso concluido}.

	\item \sistema Muestra la pantalla \Tref{UI1}{UI1 Inicio}.

	\item \finTA

\end{enumerate}


%-------------------------trayectoria Alternativa B-----------------%
\begin{large}
	\Talterna{CU4}{B}\\
\end{large}	
\textbf{Condición:} \textit{El usuario ha cambiado el periodo establecido}

\begin{enumerate}[{B-}1.]

	\item \actor \label{CU4:Dia}Presiona un botón del menú \textbf{Cambio de periodo} de la pantalla \Tref{UI4}{UI4 Resultados de consulta}.

	\item \sistema Verifica que exista al menos 1 noticia en el periodo establecido. \TAref{CU4}{C}

	\item \sistema Muestra hasta 15 noticias de las ordenadas previamente, de acuerdo a la regla de negocio \RNref{RN5}{Orden de publicación} de la sección previamente elegida (filtro de sección) y del día seleccionado en el paso \ref{CU4:Dia} (filtro de fecha), como se visualiza en la pantalla \Tref{UI5}{UI5 Cambio de periodo}.

	\item \sistema Continua en el paso \ref{CU4:Consulta} de la trayectoria principal.

	\item \finTA

\end{enumerate}


%-------------------------trayectoria Alternativa C-----------------%
\begin{large}
	\Talterna{CU4}{C}\\
\end{large}	
\textbf{Condición:} \textit{No se ha encontrado noticias en el día seleccionado \TEref{CU4}{Uno}}

\begin{enumerate}[{C-}1.]

	\item \sistema Muestra el mensaje \Tref{MSG2}{MSG2 Petición vacía}, en la pantalla \Tref{UI4}{UI4 Resultados de consulta}.

	\item \sistema Continua en el paso \ref{CU4:Consulta} de la trayectoria principal.

	\item \finTA

\end{enumerate}

