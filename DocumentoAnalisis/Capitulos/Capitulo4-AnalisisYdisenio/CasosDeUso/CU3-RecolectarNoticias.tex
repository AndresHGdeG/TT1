\Tlabel{CU3}\Tsubsection{CU3 Recolectar noticias}

%================================Intriodcucción==================================%
%----------------------Resumen-----------------------------------%
\begin{large}
	\textbf{Resumen}\\
\end{large}

Brinda un servicio para recolectar noticias en la web (Crawler); Toma como punto de partida los portales registrados en el diccionario \textbf{URL's}. Se crea un proceso de trabajo independiente por cada liga en el diccionario, para simular un ambiente de extracción en paralelo; De cada sitio se recoleta las noticas publicadas; De cada artículo se obtiene \textbf{Fecha de publicación}, \textbf{Título}, \textbf{Contenido}, \textbf{URL de la noticia}, \textbf{Nombre de la página fuente} y de contar con ello el \textbf{Resumen}. Cabe destacar que las ligas contenidas en los sitios visitdos son extraidas para suposterior análisis.\\

\begin{large}
	\textbf{Descripción}\\
\end{large} 

%=====================================Tabla 1====================================%

\begin{tabular}{|l|l|}
%-----------------------Ecanbezado-----------------------------------%
	\hline
	\multicolumn{1}{| >{\columncolor{black}}l|}{ \textcolor{myWhite}{\textbf{Caso de uso: }} }&
	\multicolumn{1}{| >{\columncolor{black}}l|}{ \textcolor{myWhite}{CU3 Recolectar noticias} }\\
	\hline
%-----------------------Actor----------------------------------------%
	\textbf{Actor:} & 	Usuario\\
	\hline

%-----------------------Propósito------------------------------------%

	\textbf{Propósito:} & Brindar una herramienta de recolección del ciber\\
	&espacio(Crawler). \\
	\hline

%----------------------Entradas--------------------------------------%

	\textbf{Entradas:} & URL de las paginas por consultar\\
	\hline

%-----------------------Salidas--------------------------------------%

	\textbf{Salidas:} &$\bullet$ Noticias; De cada una se obtiene\\
	&\ \ $\circ$ \textbf{Fecha de publicación}\\	
	&\ \ $\circ$ \textbf{Título}\\	
	&\ \ $\circ$ \textbf{Contenido}\\	
	&\ \ $\circ$ \textbf{Resumen}\\	
	&\ \ $\circ$ \textbf{URL de la noticia}\\	
	&\ \ $\circ$ \textbf{Nombre de la página fuente}\\
	&$\bullet$ \Tref{MSG7}{MSG7 Petición vacía}\\	
	&$\bullet$ \Tref{MSG9}{MSG9 Fallo en la recolección}\\
	\hline

%-----------------------Precondiciones-------------------------------%

	\textbf{Precondición:} & EL \textbf{Diccionario de URL'S} debe contener los\\
	&vínculos de los sitios a consultar\\
	\hline
%-----------------------Postcondiciones------------------------------%

	\textbf{Postcondiciones:} &$\bullet$  Las noticias recolectadas serán filtradas\\
	&\ \  por su fecha de difusión\\
	&$\bullet$ Las noticias recolectadas serán clasificadas\\
	\hline
%-----------------------Reglas de negocio----------------------------%

	\textbf{Reglas de negocio:} &$\bullet$ \RNref{RN11}{Sitios restringidos} \\
	&$\bullet$ \RNref{RN13}{Profundidad de búsqueda} \\
	&$\bullet$ \RNref{RN14}{Número de peticiones} \\
	\hline

%---------------------------Errores----------------------------------%

%------Error 1----------%	
	\textbf{Errores:}&$\bullet$ \TError{CU3}{Uno} Cuando no se ha encontrado noticias en \\
	&\ \ el periodo establecido se muestra el mensaje \Tref{MSG7}{MSG7}\\
	&\ \ \Tref{MSG7}{Petición vacía}, fin del caso de uso\\

%------Error 2----------%

	 &$\bullet$ \TError{CU3}{Dos} Cuando no se puede extraer información de\\
	 &\ \  los sitios brindados, se muestra el mensaje \Tref{MSG9}{MSG9} \\
	 &\ \ \Tref{MSG9}{Fallo en la recolección}, fin del caso de uso\\
	\hline
%-------------------------Autor--------------------------------------%
	\textbf{Autor:} & Carlos Andres Hernandez, Luis Daniel Meza\\
	\hline
\end{tabular}\\\\



%============================Trayectorias========================================%

%-----------------------Trayectoria Principal-----------------------%


\begin{large}
	\textbf{Trayectoria principal}\\
\end{large}	

\begin{enumerate}[1.]
	\item \actor Solicita realizar una búsqueda de noticias con el botón \textbf{Buscar} de la pantalla \Tref{UI2}{UI2 Sección política}.
	\item \sistema Obtiene los vínculos de los sitios web registrados en el diccionario \textbf{URL's}.
	\item \sistema 
	\item \finCU	
\end{enumerate}


%-------------------------trayectoria Alternativa A-----------------%

\begin{large}
	\textbf{Trayectoria alternativa A:}\\
\end{large}	
\textbf{Condición:} \textit{Se escribe la condición}
\begin{enumerate}[{A-}1.]

	\item \actor lorem ipsum
	\item \sistema lorem ipsum
	\item \finTA	

\end{enumerate}


%----------------------trayectoria Alternativa B-------------------%
\begin{large}
	\textbf{Trayectoria alternativa B:}\\
\end{large}	
\textbf{Condición:} \textit{Se escribe la condición}

\begin{enumerate}[{B-}1.]

	\item \actor lorem ipsum
	\item \sistema lorem ipsum
	\item \finTA	

\end{enumerate}


%================================Puntos de extención=============================%


\begin{large}
	\textbf{Puntos de extensión}\\
\end{large}	

%--------------------Puntos de extención 1------------------------%
\textbf{Causa de la extensión:} Lorem ipsum\\
\textbf{Región de la trayectorioa:} Lorem ipsum\\
\textbf{Extiende a :} Lorem ipsum\\\\

%--------------------Puntos de extención 2------------------------%
\textbf{Causa de la extensión:} Lorem ipsum\\
\textbf{Región de la trayectorioa:} Lorem ipsum\\
\textbf{Extiende a :} Lorem ipsum\\

