
  
  En esta sección se describen las reglas de negocio implementadas en el trabajo propuesto.\\\\


  %------------------RN1-----------------------%
     \DGline{RN1}{Número de palabras}
    \begin{itemize}
      \item \textbf{Tipo:}  
      \item \textbf{Descripción:}  La notica debe tener almenos 180 palabras
      \item \textbf{Ejemplo:}
      \item \textbf{Referenciado por:} nombre caso de uso \\
    \end{itemize}

  %------------------RN2-----------------------%
     \DGline{RN2}{Lenguaje de direcciones web}

    \begin{itemize}
      \item \textbf{Tipo:}  
      \item \textbf{Descripción:} Las direcciones de los sitios a consultar deben estar redactadas en lenguaje español.
      \item \textbf{Referenciado por:} \Tref{CU1}{CU1 Mostrar noticias} \\
    \end{itemize}

  %------------------RN3-----------------------%
     \DGline{RN3}{Lenguaje de noticias}

    \begin{itemize}
      \item \textbf{Tipo:}  
      \item \textbf{Descripción:} Las noticias deben estar redactadas en lenguaje español mexicano.
      \item \textbf{Ejemplo:}
      \item \textbf{Referenciado por:}  \\
    \end{itemize}

  %------------------RN4-----------------------%
     \DGline{RN4}{Extración de información}

    \begin{itemize}
      \item \textbf{Tipo:}  
      \item \textbf{Descripción:} Solo se puede recoletar información de los sitios que lo permitan.
      \item \textbf{Ejemplo:}
      \item \textbf{Referenciado por:}  \\
    \end{itemize}

    %------------------RN5-----------------------%
     \DGline{RN5}{Porcentaje de aceptación}

    \begin{itemize}
      \item \textbf{Tipo:}  
      \item \textbf{Descripción:} Solo se puede mostrar una noticia si cumple con un porcentaje de aceptación mayor a 60\%.
      \item \textbf{Ejemplo:}
      \item \textbf{Referenciado por:}  \\
    \end{itemize}

    %------------------RN6-----------------------%
     \DGline{RN6}{Fecha de consulta}

    \begin{itemize}
      \item \textbf{Tipo:}  
      \item \textbf{Descripción:} La fecha de incio debe ser menor o igual a la fecha fin de consulta.
      \item \textbf{Ejemplo:}
      \item \textbf{Referenciado por:}  \\
    \end{itemize}


    %------------------RN7-----------------------%
     \DGline{RN7}{Fecha actual}

    \begin{itemize}
      \item \textbf{Tipo:}  
      \item \textbf{Descripción:} La fecha fin de consulta no puede ser mayor a la fecha actual.
      \item \textbf{Referenciado por:}  \\
    \end{itemize}

      %------------------RN8-----------------------%
     \DGline{RN8}{Campos obligatorios}

    \begin{itemize}
      \item \textbf{Tipo:}  
      \item \textbf{Descripción:} La fecha incio y fecha fin de consulta no pueden estar vacios o incompletos.
      \item \textbf{Ejemplo:}
      \item \textbf{Referenciado por:}  \\
    \end{itemize}
      %------------------RN9-----------------------%
     \DGline{RN9}{Sitios restringidos}

    \begin{itemize}
      \item \textbf{Tipo:}m
      \item \textbf{Descripción:} No se debe acceder a las siguientes páginas:
        \begin{itemize}
            \item Facebook
            \item Youtube
            \item Twitter
            \item Instagram 
        \end{itemize} 
      \item \textbf{Ejemplo:}
      \item \textbf{Referenciado por:}  \\
    \end{itemize}

      %------------------RN10-----------------------%
     \DGline{RN10}{Fecha de partida}

    \begin{itemize}
      \item \textbf{Tipo:} Cálculo
      \item \textbf{Descripción:} La fecha inicio debe calcularse restando 5 días a la fecha actual.
      \item \textbf{Ejemplo:} 
      \item \textbf{Referenciado por:} \Tref{CU1}{CU1 Mostrar noticias} \\
    \end{itemize}

      %------------------RN11-----------------------%
     \DGline{RN11}{Orden de publicación}

    \begin{itemize}
      \item \textbf{Tipo:} Descripción
      \item \textbf{Descripción:} Las noticias se muestrán de forma descendente deacuerdo a su fecha de difusión; Es decir la primera publicación en mostrarse es aquella que tiene la fecha y hora mas sercana a la del sistema.
      \item \textbf{Ejemplo:} 
      \item \textbf{Referenciado por:} \Tref{CU1}{CU1 Mostrar noticias} \\
    \end{itemize}
