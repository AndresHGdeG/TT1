

En esta sección se describen las reglas de negocio implementadas en el trabajo propuesto.\\\\
%------------------RN1-----------------------%
\DGline{RN1}{Número de palabras}
\begin{itemize}
  \item \textbf{Tipo:} Dominio. 
  \item \textbf{Descripción:}  La notica debe tener al menos 180 palabras
%  \item \textbf{Ejemplo:}
  \item \textbf{Referenciado por: CU3 Recolectar noticias} 
\end{itemize}

%------------------RN2-----------------------%
\DGline{RN2}{Lenguaje de noticias}

\begin{itemize}
  \item \textbf{Tipo:} Dominio.  
  \item \textbf{Descripción:} Las noticias deben estar redactadas en lenguaje español.
%  \item \textbf{Ejemplo:}
  \item \textbf{Referenciado por: CU4 Clasificar noticias}  \\
\end{itemize}
%------------------RN3-----------------------%
\DGline{RN3}{Diccionario de sitios}

\begin{itemize}
  \item \textbf{Tipo:} Dominio.
  \item \textbf{Descripción:} Solo se puede recolectar información de los siguientes sitios.\\

  \begin{itemize}

    \item \textbf{El Universal}: https://www.eluniversal.com.mx/
    \item \textbf{Azteca Noticias}: https://www.aztecanoticias.com.mx/
    \item \textbf{Aristegui Noticias}: https://aristeguinoticias.com/
    \item \textbf{Excelsior}: https://www.excelsior.com.mx/
    \item \textbf{La Jornada}: https://www.jornada.com.mx/ultimas
    \item \textbf{Milenio}: https://www.milenio.com/
    \item \textbf{Yahoo}: https://es-us.noticias.yahoo.com/
    \item \textbf{Sopitas}: https://www.sopitas.com/
    \item \textbf{SDP Noticias}: https://www.sdpnoticias.com/
    \item \textbf{Uno TV}: https://www.unotv.com/inicio/

  \end{itemize} 
%  \item \textbf{Ejemplo:}
  \item \textbf{Referenciado por: CU3 Recolectar noticias}  \\
\end{itemize}
%------------------RN4-----------------------%
\DGline{RN4}{Umbral de grado de petenencia}

\begin{itemize}
  \item \textbf{Tipo:}  Flujo.
  \item \textbf{Descripción:} Solo se puede mostrar una noticia si su grado de pertenencia a una sección
  es mayor o igual al umbral establecido.
%  \item \textbf{Ejemplo:}
  \item \textbf{Referenciado por: CU4 Clasificar noticias}  \\
\end{itemize}



%------------------RN5-----------------------%
\DGline{RN5}{Orden de publicación}

\begin{itemize}
  \item \textbf{Tipo:} Estructura.
  \item \textbf{Descripción:} Las noticias se muestrán de forma descendente de acuerdo al grado de pertenencia
    a la sección.
%  \item \textbf{Ejemplo:} 
  \item \textbf{Referenciado por:} \Tref{CU2}{CU2 Buscar noticias} \\
\end{itemize}

%------------------RN6----------------------%
\DGline{RN6}{Número de noticias recolectadas}

\begin{itemize}
  \item \textbf{Tipo:} Dominio.
  \item \textbf{Descripción:} De cada sitio establecido se recolectan las noticias que se encuentren en el periodo 
    establecido (4 días)

%  \item \textbf{Ejemplo:} 
  \item \textbf{Referenciado por: CU3 Recolectar noticias} \\
\end{itemize}
