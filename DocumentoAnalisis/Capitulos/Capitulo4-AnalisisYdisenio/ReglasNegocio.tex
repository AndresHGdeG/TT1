

En esta sección se describen las reglas de negocio implementadas en el trabajo propuesto.\\\\
%------------------RN1-----------------------%
\DGline{RN1}{Número de palabras}
\begin{itemize}
  \item \textbf{Tipo:}  
  \item \textbf{Descripción:}  La notica debe tener almenos 180 palabras
  \item \textbf{Ejemplo:}
  \item \textbf{Refer:}
\end{itemize}
%------------------RN2-----------------------%
\DGline{RN2}{Lenguaje de direcciones web}

\begin{itemize}
  \item \textbf{Tipo:}  
  \item \textbf{Descripción:} Las direcciones de los sitios a consultar deben estar redactadas en lenguaje español.
  \item \textbf{Referenciado por:} \Tref{CU1}{CU1 Seleccionar sección} \\
\end{itemize}
%------------------RN3-----------------------%
\DGline{RN3}{Lenguaje de noticias}

\begin{itemize}
  \item \textbf{Tipo:}  
  \item \textbf{Descripción:} Las noticias deben estar redactadas en lenguaje español mexicano.
  \item \textbf{Ejemplo:}
  \item \textbf{Referenciado por:}  \\
\end{itemize}
%------------------RN4-----------------------%
\DGline{RN4}{Restricción en la recoleción}

\begin{itemize}
  \item \textbf{Tipo:}  
  \item \textbf{Descripción:} Solo se puede recoletar información de los sitios que lo permitan.
  \item \textbf{Ejemplo:}
  \item \textbf{Referenciado por:}  \\
\end{itemize}
%------------------RN5-----------------------%
\DGline{RN5}{Porcentaje de aceptación}

\begin{itemize}
  \item \textbf{Tipo:}  
  \item \textbf{Descripción:} Solo se puede mostrar una noticia si cumple con un porcentaje de aceptación mayor a 60\%.
  \item \textbf{Ejemplo:}
  \item \textbf{Referenciado por:}  \\
\end{itemize}
%------------------RN6-----------------------%
\DGline{RN6}{Formato de fecha}

\begin{itemize}
  \item \textbf{Tipo:} Derivación.
  \item \textbf{Descripción:} El formato de fecha se define como:\\

  $F=D/M/A$

  $donde$\\ 

  $D=\{x/x\ E\ N,1\leq\ x\ \leq31\}$\\
  $M=\{y/y\ E\ N,1\leq\ y\ \leq12\}$\\
  $A=\{z/z\ E\ N,1990\leq\ z\ \leq\Lambda_i\}$\\

  $F:Fecha$\\
  $\Lambda_i:Anio\_actual$\\

  \item \textbf{Ejemplo:} 
  \item \textbf{Referenciado por:} \Tref{CU2}{CU2 Buscar noticias} \\
\end{itemize}

%------------------RN7-----------------------%
\DGline{RN7}{Perido preestablecido}

\begin{itemize}
  \item \textbf{Tipo:} Cálculo
  \item \textbf{Descripción:} El formato de fecha es el descrito en la \RNref{RN6}{Formato de fecha}; El periodo establecido se define de la siguiente forma.\\\\
  \textbf{Fecha fin}:Toma el valor de la fecha actual.\\
  \textbf{Fecha inicio}: Es colocada 5 día antes de la fecha actual; Se muestra la forma completa para el cálculo del día, mes y año:\\
  $Sea$\\
  $D_a:Dia\_actual$\\
  $M_a:Mes\_actual$\\
  $A_a:Anio\_actual$\\
  $F_i:Fecha\_inicio$\\
  $mod:Operacion\ modulo$\\
  $\Psi(M_a):Funcion\ dias\ de \ mes$\\

  $\xi=\frac{D_a-5}{\mid D_a-5 \mid}$\\

  $\delta=\frac{\xi(\mid\xi-1\mid)}{2}$\\

  La fecha toma el sigueinte valor:\\
  $F_i=D_c/M_c/A_c$\\ 
  Existen 4 casos para calcular el día, mes y años; Dependen del mes y el día actual: \\

  \begin{tabular}{|l|c|c|}
  	\hline
	\multicolumn{1}{| >{\columncolor{black}}l|}{ \textcolor{myWhite}{\textbf{Caso:}} }&
	\multicolumn{1}{| >{\columncolor{black}}c|}{ \textcolor{myWhite}{1} }&\multicolumn{1}{| >{\columncolor{black}}c|}{ \textcolor{myWhite}{2} }\\
	\hline
	\textbf{Restricción:}&$2\leq\ M_a\leq\ 12;D_a\neq5$  &$2\leq\ M_a\leq\ 12;D_a=5$\\
	\hline 
	\textbf{$D_c:$}&$(D_a-5)\pmod{\Psi(M_a-1)+1}$   &$\Psi(M_a-1)$\\
	\hline
	\textbf{$M_c:$}&$M_a+\delta$        			 &$M_a-1$\\
	\hline
	\textbf{$A_c:$}&$A_a$        				     &$A_a$\\
	\hline
  \end{tabular}


  \begin{tabular}{|l|c|c|}
  	\hline
	\multicolumn{1}{| >{\columncolor{black}}l|}{ \textcolor{myWhite}{\textbf{Caso:}} }&
	\multicolumn{1}{| >{\columncolor{black}}c|}{ \textcolor{myWhite}{3} }&\multicolumn{1}{| >{\columncolor{black}}c|}{ \textcolor{myWhite}{4} }\\
	\hline
	\textbf{Restricción:}&$M_a=1;D_a \neq 5$&\ \ \ \ \ \ $M_a=1;D_a=5$\ \ \ \ \ \\
	\hline 
	\textbf{$D_c:$}&\ \ \ \ \ \ \ \ \ $(D_a-5)\pmod{32}$\ \ \ \ \ \ \ \ & $31$\\
	\hline
	\textbf{$M_c:$}&$\xi\pmod{13}$&$12$\\
	\hline
	\textbf{$A_c:$}&$A_a+\delta$&$A_a-1$\\
	\hline
  \end{tabular}


  \item \textbf{Ejemplo:} 
  \item \textbf{Referenciado por:} \Tref{CU1}{CU1 Seleccionar sección} \\
\end{itemize}%------------------RN8-----------------------%
\DGline{RN8}{Periodo válido}

\begin{itemize}
  \item \textbf{Tipo:} Derivación.
  \item \textbf{Descripción:}\\
  $Sea$\\

  $F_i:Fecha\_inicio$\\
  $F_f:Fecha\_fin$\\

  $T=\{Valido,Invalido\}$\\
  $\psi\ \epsilon\ T$\\

  Un periodo de fecha se defino como:\\

  $(\psi=Valido)\leftrightarrow(F_i\ \leq\ F_f)$\\
  $(\psi=Invalido)\leftrightarrow(F_i\ >\ F_f)$\\
  \item \textbf{Ejemplo:}
  \item \textbf{Referenciado por:} \Tref{CU2}{CU2 Buscar noticias} \\
\end{itemize}
%------------------RN9-----------------------%
\DGline{RN9}{Limite de periodo}

\begin{itemize}
  \item \textbf{Tipo:} Derivación. 
  \item \textbf{Descripción:} 

  $Sea$\\

  $F_i:Fecha\_inicio$\\
  $F_f:Fecha\_fin$\\
  $F_a:Fecha\_actual$\\
  $F_c:01/01/1990$\\

  $T=\{Valido,Invalido\}$\\
  $\psi\ \epsilon\ T$\\

  $\Phi=[F_c,F_a]$\\
  $I=[F_i,F_f]$\\
  

  Un intervalo de tiempo dentro de los limites del sistema se define como:\\

  $(\psi=Valido)\leftrightarrow(I\ \subseteq\ \Phi)$\\
  $(\psi=Invalido)\leftrightarrow(I\ \nsubseteq \Phi)$\\

  \item \textbf{Referenciado por:} \Tref{CU2}{CU2 Buscar noticias} \\
\end{itemize}

%------------------RN10-----------------------%
\DGline{RN10}{Campos obligatorios}

\begin{itemize}
  \item \textbf{Tipo:} Restricción.  
  \item \textbf{Descripción:} Los campos marcados con \* no se deben omitir.
  \item \textbf{Ejemplo:} 
  \item \textbf{Referenciado por:}  \\
\end{itemize}
%------------------RN11-----------------------%
\DGline{RN11}{Sitios restringidos}

\begin{itemize}
  \item \textbf{Tipo:}m
  \item \textbf{Descripción:} No se debe acceder a las siguientes páginas:
    \begin{itemize}
        \item Facebook
        \item Youtube
        \item Twitter
        \item Instagram 
    \end{itemize} 
  \item \textbf{Ejemplo:}
  \item \textbf{Referenciado por:}  \\
\end{itemize}
%------------------RN12-----------------------%
\DGline{RN12}{Orden de publicación}

\begin{itemize}
  \item \textbf{Tipo:} Descripción
  \item \textbf{Descripción:} Las noticias se muestrán de forma descendente deacuerdo a su fecha de difusión; Es decir la primera publicación en mostrarse es aquella que tiene la fecha y hora mas sercana a la del sistema.
  \item \textbf{Ejemplo:} 
  \item \textbf{Referenciado por:} \Tref{CU2}{CU2 Buscar noticias} \\
\end{itemize}


