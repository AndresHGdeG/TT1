
  
  En esta sección se describen las reglas de negocio implementadas en el trabajo propuesto.\\\\


  %------------------RN1-----------------------%
     \DGline{RN1}{Número de palabras}
    \begin{itemize}
      \item \textbf{Tipo:}  
      \item \textbf{Descripción:}  La notica debe tener almenos 180 palabras
      \item \textbf{Ejemplo:}
      \item \textbf{Referenciado por:} nombre caso de uso \\
    \end{itemize}

  %------------------RN2-----------------------%
     \DGline{RN2}{Lenguaje de direcciones web}

    \begin{itemize}
      \item \textbf{Tipo:}  
      \item \textbf{Descripción:} Las direcciones de los sitios a consultar deben estar redactadas en lenguaje español.
      \item \textbf{Referenciado por:} \Tref{CU1}{CU1 Seleccionar sección} \\
    \end{itemize}

  %------------------RN3-----------------------%
     \DGline{RN3}{Lenguaje de noticias}

    \begin{itemize}
      \item \textbf{Tipo:}  
      \item \textbf{Descripción:} Las noticias deben estar redactadas en lenguaje español mexicano.
      \item \textbf{Ejemplo:}
      \item \textbf{Referenciado por:}  \\
    \end{itemize}

  %------------------RN4-----------------------%
     \DGline{RN4}{Extración de información}

    \begin{itemize}
      \item \textbf{Tipo:}  
      \item \textbf{Descripción:} Solo se puede recoletar información de los sitios que lo permitan.
      \item \textbf{Ejemplo:}
      \item \textbf{Referenciado por:}  \\
    \end{itemize}

    %------------------RN5-----------------------%
     \DGline{RN5}{Porcentaje de aceptación}

    \begin{itemize}
      \item \textbf{Tipo:}  
      \item \textbf{Descripción:} Solo se puede mostrar una noticia si cumple con un porcentaje de aceptación mayor a 60\%.
      \item \textbf{Ejemplo:}
      \item \textbf{Referenciado por:}  \\
    \end{itemize}

    %------------------RN6-----------------------%
     \DGline{RN6}{Periodo válido}

    \begin{itemize}
      \item \textbf{Tipo:} Derivación.
      \item \textbf{Descripción:}\\
      $Sea$\\

      $F_i:Fecha\_inicio$\\
      $F_f:Fecha\_fin$\\

      $T=\{Valido,Invalido\}$\\
      $\psi\ \epsilon\ T$\\

      Un periodo de fecha válido se defino como:\\

      $(\psi=Valido)\leftrightarrow(F_i\ \leq\ F_f)$\\
      $(\psi=Invalido)\leftrightarrow(F_i\ >\ F_f)$\\
      \item \textbf{Ejemplo:}
      \item \textbf{Referenciado por:} \Tref{CU2}{CU2 Buscar noticias} \\
    \end{itemize}


    %------------------RN7-----------------------%
     \DGline{RN7}{Limite de periodo}

    \begin{itemize}
      \item \textbf{Tipo:} Derivación. 
      \item \textbf{Descripción:} 

      $Sea$\\

      $F_i:Fecha\_inicio$\\
      $F_f:Fecha\_fin$\\
      $F_a:Fecha\_actual$\\
      $F_c:01/01/1990$\\

      $T=\{Valido,Invalido\}$\\
      $\psi\ \epsilon\ T$\\

      $\Phi=[F_c,F_a]$\\
      $I=[F_i,F_f]$\\
      

      Un intervalo de tiempo válido dentro de los limites del sistema se define como:\\

      $(\psi=Valido)\leftrightarrow(I\ \subseteq\ \Phi)$\\
      $(\psi=Invalido)\leftrightarrow(I\ \nsubseteq \Phi)$\\

      \item \textbf{Referenciado por:} \Tref{CU2}{CU2 Buscar noticias} \\
    \end{itemize}

      %------------------RN8-----------------------%
     \DGline{RN8}{Campos obligatorios}

    \begin{itemize}
      \item \textbf{Tipo:} Restricción.  
      \item \textbf{Descripción:} Los campos marcados con \* no se deben omitir.
      \item \textbf{Ejemplo:} 
      \item \textbf{Referenciado por:}  \\
    \end{itemize}
      %------------------RN9-----------------------%
     \DGline{RN9}{Sitios restringidos}

    \begin{itemize}
      \item \textbf{Tipo:}m
      \item \textbf{Descripción:} No se debe acceder a las siguientes páginas:
        \begin{itemize}
            \item Facebook
            \item Youtube
            \item Twitter
            \item Instagram 
        \end{itemize} 
      \item \textbf{Ejemplo:}
      \item \textbf{Referenciado por:}  \\
    \end{itemize}

      %------------------RN10-----------------------%
     \DGline{RN10}{Fecha de partida}

    \begin{itemize}
      \item \textbf{Tipo:} Cálculo
      \item \textbf{Descripción:} La fecha inicio debe calcularse restando 5 días a la fecha actual.
      \item \textbf{Ejemplo:} 
      \item \textbf{Referenciado por:} \Tref{CU1}{CU1 Seleccionar sección} \\
    \end{itemize}

      %------------------RN11-----------------------%
     \DGline{RN11}{Orden de publicación}

    \begin{itemize}
      \item \textbf{Tipo:} Descripción
      \item \textbf{Descripción:} Las noticias se muestrán de forma descendente deacuerdo a su fecha de difusión; Es decir la primera publicación en mostrarse es aquella que tiene la fecha y hora mas sercana a la del sistema.
      \item \textbf{Ejemplo:} 
      \item \textbf{Referenciado por:} \Tref{CU2}{CU2 Buscar noticias} \\
    \end{itemize}

      %------------------RN12-----------------------%
     \DGline{RN12}{Formato de fecha}

    \begin{itemize}
      \item \textbf{Tipo:} Derivación.
      \item \textbf{Descripción:} Un formato de fecha correcto se define como:\\

      $F=D/M/A$

      $donde$\\ 

      $F:Fecha$\\
      $\Lambda_i:Anio\_actual$\\

      $D=\{x/x\ E\ N,1\leq\ x\ \leq31\}$\\
      $M=\{y/y\ E\ N,1\leq\ y\ \leq12\}$\\
      $A=\{z/z\ E\ N,1990\leq\ z\ \leq\Lambda_i\}$\\




      \item \textbf{Ejemplo:} 
      \item \textbf{Referenciado por:} \Tref{CU2}{CU2 Buscar noticias} \\
    \end{itemize}
