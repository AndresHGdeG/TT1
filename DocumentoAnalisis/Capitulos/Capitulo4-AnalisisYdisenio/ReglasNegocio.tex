

En esta sección se describen las reglas de negocio implementadas en el trabajo propuesto.\\\\
%------------------RN1-----------------------%
\DGline{RN1}{Número de palabras}
\begin{itemize}
  \item \textbf{Tipo:}  
  \item \textbf{Descripción:}  La notica debe tener almenos 180 palabras
%  \item \textbf{Ejemplo:}
  \item \textbf{Refer:CU3 Recolectar noticias} 
\end{itemize}

%------------------RN2-----------------------%
\DGline{RN2}{Lenguaje de noticias}

\begin{itemize}
  \item \textbf{Tipo:}  
  \item \textbf{Descripción:} Las noticias deben estar redactadas en lenguaje español.
%  \item \textbf{Ejemplo:}
  \item \textbf{Referenciado por: CU4 Clasificar noticias}  \\
\end{itemize}
%------------------RN3-----------------------%
\DGline{RN3}{Diccionario de sistios}

\begin{itemize}
  \item \textbf{Tipo:}  
  \item \textbf{Descripción:} Solo se puede recolectar información de los siguientes sitios.\\

  \begin{itemize}

    \item \textbf{Excelsior}: https:\/\/www.excelsior.com.mx
    \item \textbf{La jornada}: https:\/\/www.jornada.com.mx
    \item \textbf{Uno noticias}: https:\/\/www.unotv.com
    \item \textbf{El Universal}: https:\/\/www.eluniversal.com.mx
    \item \textbf{Aristegui noticias}: https:\/\/aristeguinoticias.com
    \item \textbf{Medio tiempo}: https:\/\/www.mediotiempo.com
    \item \textbf{Milenio}: https:\/\/www.milenio.com 
    \item \textbf{El economista}: https:\/\/www.eleconomista.com.mx

  \end{itemize} 
%  \item \textbf{Ejemplo:}
  \item \textbf{Referenciado por:CU3 Recolectar noticias}  \\
\end{itemize}
%------------------RN4-----------------------%
\DGline{RN4}{Porcentaje de aceptación}

\begin{itemize}
  \item \textbf{Tipo:}  
  \item \textbf{Descripción:} Solo se puede mostrar una noticia si cumple con un porcentaje de aceptación mayor a 60\%.
%  \item \textbf{Ejemplo:}
  \item \textbf{Referenciado por: CU4 Clasificar noticias}  \\
\end{itemize}



%------------------RN5-----------------------%
\DGline{RN5}{Orden de publicación}

\begin{itemize}
  \item \textbf{Tipo:} Descripción
  \item \textbf{Descripción:} Las noticias se muestrán de forma descendente deacuerdo a su fecha de difusión; Es decir la primera publicación en mostrarse es aquella que tiene la fecha y hora mas cercana a la del sistema.
%  \item \textbf{Ejemplo:} 
  \item \textbf{Referenciado por:} \Tref{CU2}{CU2 Buscar noticias} \\
\end{itemize}

%------------------RN6----------------------%
\DGline{RN6}{Profundidad de búsqueda}

\begin{itemize}
  \item \textbf{Tipo:} Cálculo.
  \item \textbf{Descripción:} Se muestra la forma correcta de realizar el cáculo para obtener la profundidad de búsqueda asociada al número de noticias por recolectar, y la cantidad de hiperbínculos que cáda página contiene.\\
 
  $Sea$\\

  $\Lambda(\mu,\psi)=\log_{\psi}{(\mu(\psi-1)+\psi)}-1$\\

  $donde$\\
  $\psi:Numero\ de\ URL\ por\ pagina$\\
  $\mu:Numero\ de\ noticias\ recolectadas$\\
  $\Lambda:Profundidad\ de\ busqueda$\\



%  \item \textbf{Ejemplo:} 
  \item \textbf{Referenciado por: CU3 Recolectar noticias} \\
\end{itemize}

%------------------RN7-----------------------%
\DGline{RN7}{Número de peticiones}

\begin{itemize}
  \item \textbf{Tipo:} Restricción
  \item \textbf{Descripción:} El número de peticiones realizadas a una página no debe exceder el permitido por el dominio.
%  \item \textbf{Ejemplo:} 
  \item \textbf{Referenciado por: CU3 Recolectar noticias} \\
\end{itemize}
