\TChapter{Avances}{epsilon}
\ \\\\
En esta sección se mostrarán los avances que hasta el día de hoy se tienen.

\section{Herramientas utilizadas}
Para el desarrollo de los avances para el presente trabajo se utilizó el lenguaje de programación Python, se utilizó Anaconda debido a que
es una distribucion de Python que se utiliza para el análisis de datos, es decir con Anaconda tenemos un ambiente de trabajo para el análisis 
de datos. Mientras que para el web- crawling se utilizó Scrapy, el cual es un framework que nos permite el web scraping para la extracción de 
información de los sitios web.

\section{Estudio previo}
Una vez que se eligieron las herramientas que nos iban a proporcionar la información de las páginas web, se procedió con el estudió de los 
sitios de los cuales se extraerían noticias para el entrenamiento.

En primera instancia se eligiero 9 sitios web (\ref{tabla:sitios}):

\begin{itemize}
    \item 3 Sitios de foros de noticias: Aristegui Noticias, SDP Noticias y Sopitas.
    \item 3 Sitios de diarios: El Universal, La Jornada y Excelsior.
    \item 3 Sitios televisivos: TV Azteca, Televisa y Once Noticias.
\end{itemize}


\begin{table}[htbp]
	\begin{center}
	\begin{tabular}{|lrcccccccc|}
		\hline
		Sección  & El            & La           & Excelsior     & Aristegui             & SDP           & Sopitas   & Azteca            & Televisa      & Once  \\ 
                 & Universa1     & Jornada      &               & Noticias              & Noticias      &           & Noticias          &               & Noticias      \\ \hline
        Nacional & Nacional      & -            & Nacional      & México                & Nacional      & Noticias  & -                 & Nacional      & Nacional      \\ \hline
        Internacional & Mundo    & Mundo        & Global        & Destacado/Mundo       & Internacional & -         & Internacional     & Internacional & Internacional \\ \hline
        Ciudad   & Metrópoli     & Capital      & Comunidad     & -                     & CDMX          & -         & -                 & CDMX          & CDMX      \\ \hline
        Estados  & Estados       & Estados      & Estados       & Sociedad/México       & Locales       & -         & Estados           & Estados       & Nacional   \\ \hline
        Economía & Cartera       & Economía     & Dinero        & Economía              & Economía      & -         & Finanzas          & Economía      & Economía   \\ \hline
        Deportes & Deportes      & Deportes     & Adrenalina    & Deportes              & Deportes      & Deportes  & -                 & Deportes      & Deportes   \\ \hline
        Espectáculos & Espectáculos & Espectáculos & Función    & -                     & En el show    & Entretenimiento & -           & Espectáculos  & Deportes    \\ \hline
        Cultura  & Cultura       & Cultura      & -             & -                     & -             & -         & -                 & Arte y cultura & Cultura   \\ \hline
        Política & -             & Política     & Política      & Poderes               & -             & -         & Política          & Política      & -   \\ \hline
        
		\hline
	\end{tabular}
	\caption{Tabla de las secciones de los sitios web definidos}
	\label{tabla:sitios}
	\end{center}
\end{table}

Una vez que se analizarón las secciones con las que contaba cada sitio se procedió a homologar las secciones en las cuales la mayoría de los 
sitios coincidian, por lo cual se quedarón definidas 5 secciones para clasificación de las noticias extraídas.

\begin{itemize}
    \item Política
    \item Deportes
    \item Ciencia y tecnología
    \item Economía
    \item Cultura
\end{itemize}

Posteriormente se procedio con la recolección de noticias. 
La biblioteca utilizada es Scrapy la cual nos servirá para el Scrapeo.

Una de las cosas que se debía investigar sobre cada sitio web fue su estructura XML, al saber la estructura del contenido de una página  
nos permite realizar el Scrapeo de manera correcta y eso depende de cada sitio, ya que no todos los sitios cuentan con una página la cual tenga 
las mismas secciones o el mismo orden.
\\
