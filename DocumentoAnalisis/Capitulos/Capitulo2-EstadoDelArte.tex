
\chapter{Estado del arte}

\section{Introducción}

	A continuación, se mostrarán distintos trabajos nacionales e internacionales, así como herramientas las cuales desempeñan una labor similar al propuesto en este trabajo.\\

\section{Trabajos nacionales}%[pdf]

	El trabajo \textit{Clasificación Automática de Textos de Desastres Naturales en México}
	propone clasificar noticias del ámbito en Desastres Naturales, utilizando estrategias de reducción de dimensionalidad conocidas como umbral en la frecuencia y ganancia en la información, los métodos de clasificación utilizados fueron el clasificador simple de Bayes y vecinos más cercanos.\\

	Se utilizaron 375 noticias del periódico \"Reforma\" como conjunto de entrenamiento, para posteriormente clasificarlas (relevantes e irrelevantes), de los cuales el 11.5\% de noticias eran relevantes y el 88.5\% restante eran irrelevantes.\\

	Una vez obtenido el conjunto de noticias se procedió con un pre\-procesamiento, el cual reducía el tamaño de los documentos, eliminando las partes de los textos que no se consideraban relevantes; posteriormente se realizó el indexado, el cual los documentos son representdos por vectores de palabras en un espacio de dimensionalidad \textit{n} en el cual se logró una reducción de dimensionalidad en donde finalemente se utilizaron técnicas de clasificación como el algoritmo simple de Bayes en el cual se obtuvo un resultado del 97\% de efectividad al clasifciar noticias de desastres naturales.\\



\section{Trabajos internacionales}
	%https://scielo.conicyt.cl/scielo.php?script=sci_arttext&pid=S0718-09342014000300001
	La obra \textit{Clasificación Automática de Textos Usando Redes de Palabras} propone un algoritmo para la clasificación automática de textos basado en una representación y clasificación distinta utilizada en los algoritmos de clasificación supervisada, utilizando redes de palabras.\\

	Se utilizaron 1000 mensajes de texto de la plataforma Twitter, en el idioma español y correspondiente a distintos contextos, para posteriormente clasificar el tipo de contenido de los mensajes (positivos, negativos y neutrales), se definió un grafo como aquella red de palabras co\-cocurrentes construida a partir de un conjunto de textos clasificados; para su realización el primer proceso es llevar distintas variantes de una misma palabra a su raíz, esto para reducir la variabilidad del lenguaje posteriormente se considera las palabras plurales (terminadas con ‘s’ o ‘es’). A estas se les elimina el sufijo para compararlas con su equivalente singular, realizando el cambio de manera automática.
	Los resultados mostraron que el clasificador presenta un 80\% cercanía respecto a la clasificación realizada por una persona; su nivel de desempeño fue mayor al obtenido con el algoritmo Naive Bayes.\\


	El trabajo \textit{Document Classification for Newspaper Articles} se ha enfocado en clasificar articulos del MIT (Massachusetts Institute of Technology) de las siguientes categorías:

	\begin{itemize}
		\item Arts
		\item Features
		\item News
		\item Opinion
		\item Sports
		\item World
	\end{itemize}

	Para los cuales utilizarón el algoritmos de clasificación como el \textit{Naive Bayes} ya que era uno de los clasificadores más simples y eficaces que otras técnicas de clasificación, de igual manera utilizarón la clasificación máxima de entropia el cual provee segmentación de texto, modelado de lenguaje.
	Se utilizó un corpus de 3000 artículos en total, siendo 500 artículos de cada sección mencionada. Para el entrenamiento se utilizaron 120 artículos siendo 20 de cada sección y teniendo como resultado un 77\% de exactitud.\\

\section{Herramientas disponibles}

	Entre las herramientas utilizadas para el procesamiento de lenguaje natural y aprendizaje automático se encuentran:

	%https://cloud.google.com/natural-language/?hl=Es-419
	\begin{itemize}
		\item \textit{Google Cloud Natural Language}; Ha revelado la estructura y el significado del texto con modelos potentes de aprendizaje automático previamente entrenados en una API de REST fácil de usar y con modelos personalizados se puede utilizar para extraer información sobre personas, lugares, eventos y muchos otros datos, que se mencionan en documentos de texto, artículos periodísticos o entradas de blog. También se puede utilizar para comprender las opiniones sobre sus productos expresadas en los medios sociales o analizar la intención en las conversaciones de los clientes que se den en un centro de atención telefónica o una aplicación de mensajería[]
		\item Algoritmo Naive Bayes
		\item Procesamiento de lenguaje natural
		\item Árbol de decisión
		\item Clasificación máxima de entropía
	\end{itemize}
