
\chapter{Estado del arte}

\section{Introducción}

A continuación, se mostrarán distintos trabajos nacionales e internacionales, así como herramientas las cuales desempeñan un trabajo similar al propuesto.

\section{Trabajos nacionales}

\begin{large}
	 \textbf{Buscar uno relacionado}\\
\end{large}

Texto.\\

\begin{large}
	 \textbf{News article classification of mexican newspapers}\\
\end{large}

Artículos en periódicos son dividos en secciones como, cultura, política y deportes para ayudar al lector a encontrar información de una forma fácil. Los editores de la prensa leen los artículos y deciden cual publicar y la sección a la cual pertenece. Este documento presenta métodos de aprendizaje automático supervisado, para clasificar automaticamente noticias en secciones. Para realizar esta tarea se han recolectado 4,027 artículos junto con su sección correspondiente de tres periódicos mexicanos duranta un periodo de 6 meses. Diferentes caracteristicas fueron extraidas y un conjunto de métodos de aprendizaje fueron probados. Los resultados obtenidos muestran una precisión de 80\% en la clasificación de los artículos en su correspondiente sección de los tres periódicos seleccionados.\\



\section{Trabajos internacionales}


\begin{large}
	 \textbf{Clasificación Automática de Textos Usando Redes de Palabras}\\
\end{large}

En este trabajo se propone un algoritmo para la clasificación automática de textos basado en una representación y clasificación distinta utilizada en los algoritmos de clasificación supervisada, utilizando redes de palabras.
Se utilizaron 1000 mensajes de texto de la plataforma Twitter, en el idioma español y correspondiente a distintos contextos, para posteriormente clasificar el tipo de contenido de los mensajes (positivos, negativos y neutrales), se definió un grafo como aquella red de palabras co-cocurrentes construida a partir de un conjunto de textos clasificados; para su realización el primero de estos procesos es llevar distintas variantes de una misma palabra a su raíz, esto para reducir la variabilidad del lenguaje posteriormente se considera las palabras plurales (terminadas con ‘s’ o ‘es’). A estas se les elimina el sufijo para compararlas con su equivalente singular, realizando el cambio de manera automática.
Los resultados mostraron que el clasificador presenta un 80\% cercanía respecto a la clasificación realizada por una persona; su nivel de desempeño fue mayor al obtenido con el algoritmo Naive Bayes [6].\\


\begin{large}
	 \textbf{Document classification for newspaper articles}\\
\end{large}

En este trabajo se enfocaron en clasificar articulos del MIT (Massachusetts Institute of Technology) de las siguientes categorías:

	\begin{itemize}
		\item Arts
		\item Features
		\item News
		\item Opinion
		\item Sports
		\item World
	\end{itemize}
Para los cuales utilizarón el algoritmos de clasificación como el Naive Bayes ya que era uno de los clasificadores más simples y eficaces que otras técnicas de clasificación, de igual manera utilizarón la clasificación máxima de entropia el cual provee segmentación de texto, modelado de lenguaje.
Se utilizarón un corpus 3000 artículos en total, siendo 500 artículos de cada sección mencionada, para el entrenamiento se utilizarón 120 artículos siendo 20 de cada sección y teniendo como resultado un 77\% de exactitud [7].


\section{Herramientas disponibles}


Entre las herramientas de trabajo que son de utilidad para el procesamiento de lenguaje natural y aprendizaje automático se encuentran:\\

\begin{large}
	 \textbf{Cloud natural language}\\
\end{large}

Google Cloud Natural Language [8] revela la estructura y el significado del texto con modelos potentes de aprendizaje automático previamente entrenados en una API de REST fácil de usar y con modelos personalizados se puede utilizar para extraer información sobre personas, lugares, eventos y muchos otros datos, que se mencionan en documentos de texto, artículos periodísticos o entradas de blog. También se puede utilizar para comprender las opiniones sobre los productos expresadas en los medios sociales o analizar la intención en las conversaciones de los clientes que se den en un centro de atención telefónica o una aplicación de mensajería.\\

\begin{large}
	 \textbf{Googlebot}\\
\end{large}

Es el crawler diseñado por Google para indexar el contenido nuevo o actualizado de Internet.
Googlebot [9] no sólo tiene la capacidad de rastrear e indexar los sitios web de Internet, sino que además puede extraer información de ficheros como pueden ser PDF, XLS, DOC, etc.
Una vez el contenido está indexado, el servidor lo clasifica y establece un orden de relevancia para las distintas búsquedas que pueda efectuar un usuario, es decir, lo posiciona.\\



\begin{large}
	 \textbf{Watson natural language classifier}\\
\end{large}

Watson NLC [10] aplica técnicas de computación cognitiva para analizar un texto y proporcionar la clase que mejor encaja entre un conjunto de clases predefinidas a partir de un texto corto.
Al ser un clasificador, esta compuesto de ciertos pasos, en primera instancia se necesitan de clases las cuales son etiquetas que identificarán el texto analizado y será la salida proporcionada por el clasificador; posteriormente se debe tomar en cuenta que se necesita de una colección de textos, los cuales proporcionarán apoyo para que el clasificador logre identificar las clases ingresadas posteriormente teniendo todos estos datos se logra entrenar al clasificador, el cual proporcionará una salida dependiendo a los datos que fueron utilizados.


