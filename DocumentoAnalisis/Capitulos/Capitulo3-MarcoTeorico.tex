\chapter{Marco teórico}

En esta sección se expondrán de manera detallada conceptos los cuales son esenciales para la elaboración de este trabajo.


\section{Lenguaje}

%--------------

\section{Base de datos}

%------------------------
\section{Crawler}

Un crawler es una herramienta la cual analiza sitios web, permitiendo recolectar las páginas web para así posteriormente extraer la información que contengan. Un crawler también conocido como como robot o spider, es un sistema para la descarga masiva de páginas web. Son uno de los componentes principales de los motores de búsqueda web, los sistemas que reúnen un conjunto de páginas web, las indexan y permiten a los usuarios realizar consultas contra el índice y encontrar las páginas web que coincidan con las consultas.[\*]

%* http://infolab.stanford.edu/~olston/publications/crawling_survey.pdf

\section{Sitios web}

Un sitio web es un conjunto de páginas web

\section{Página web}

Una página web es un documento electrónico el cual forma parte de la WWWW (World Wide Web) generalmente construido en el lenguaje HTML (Hyper Text Markup Language). Este documento puede contener enlaces que nos direcciona a otra página web. Para visualizar una página web es necesario de un browser o un navegador[+]. Dentro de las páginas web podemos encontrar un sinfin de sitios los cuales pueden ser de nuestro interés.
%+http://www.madrid.org/cs/StaticFiles/Emprendedores/GuiaEmprendedor/tema7/F49_7.9_WEB.pdf

\section{Blog}

Un blog es una página web en la cual el usuario no necesita conocimientos específicos del medio electrónico ni del formato digital para poder aportar contenidos de forma inmediata, ágil y constante desde cualquier punto de conexión a Internet [\:]. En un blog el usuario puede compartir cualquier tipo de información que sea de su agrado, teniendo una mayor libertad de expresión lo cual permite que otras personas compartan y comenten su manera de expresarse.
%:http://openaccess.uoc.edu/webapps/o2/bitstream/10609/17821/5/XX08_93006_01331-3.pdf

\section{Foro}

Un foro es una herramienta de comunicación asíncrona. Los  foros permiten la comunicación de los participantes desde cualquier lugar en el que  esté  disponible  una  conexión  a Internet  sin  que  éstos  tengan  que  estar dentro del sistema al mismo tiempo, de ahí su naturaleza asíncrona [\=]. Brindando una mayor interacción entre distintos participantes y permitiendo conocer la opinión sobre un tema de distintas personas.
%= http://www.uls.edu.sv/pdf/manuales_moodle/foros.pdf


\section{Lenguaje}

El lenguaje es un medio de comunicación a traves de de un sistema de símbolos[?].
La Real Academía Española define al lenguaje como la facultad del ser humano de expresarse y comunicarse con los demás a través del sonido articulado o de otros sistemas de signos.
%? http://www.paidopsiquiatria.cat/files/12_trastornos_desarollo_lenguaje_comunicacion.pdf

\section{Procesamiento de lenguaje natural}

El procesamiento de lenguaje natural es una disciplina de la Inteligencia Artificial que se ocupa de la formulación e investigación de mecanismos computacionales para la comunicación entre persoonas y maquinas mediante el uso de Lenguajes Naturales[\'].
%' http://www.cs.us.es/cursos/ia2/temas/tema-06.pdf

\section{Corpus}

Se le llama corpues a la recopilación de un conjunto de textos, de materiales escritos y/o hablados, agrupados bajo un conjunto de criterios mínimos, para realizar ciertos análisis lingüísticos.
